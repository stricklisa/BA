% ----------------------------------------------------------------------------------------------------------
% Kapitel
% ----------------------------------------------------------------------------------------------------------
\section{Uebersicht über Ernährungserhebungsmethoden}

Ernärungserhebungsmethoden lassen sich in indirekte und direkte Methoden einteilen. Die indirekte Erfassung der Ernährung basiert auf der Sammlung statistischer Daten von Populationen oder Teilpopulationen, wie z.B. Agrarstatistiken, Einkommens- und Verbrauchsstatistik. Durch diese Zahlen können u.a. Ernährungssituationen verschiedener Länder gegenübergestellt werden \cite{muller2007ernahrungsmedizinische}. Direkte Methoden erfassen das individuelle Ernährungsverhalten einzelner Menschen. Sie werden nochmals in prospektive und retrospektive Methoden unterteilt. 


\subsection{Ziele von Ernährungserhebung}

Ernährungserhebungen werden durchgeführt, um den Lebensmittelverzehr von einzelnen pder mehrerer Personen zu erfassen. Wichtig ist die Erhebung im Gebiet der Ernährungsberatung und für wissenschaftliche Studien. Durch die Erhebung und auswertung des Lebensmittelverzehrs können Zusammenhänge zwischen der Ernährung und Erkrankungen erkannt werden. Das Bewusstsein über die Verbindung zwischen Ernährungsverhalten und der Entstehung von Krankheiten steigt stetig an. Die Ernährung gilt neben erhöhtem Blutdruck, hohen Blutfettwerten und Übergewicht zu den bedeutensten Risikofaktoren chronischer Erkrankungen \cite{hurrelmann2004einfuhrung}.

\subsection{Traditionelle direkte Ernährungserhebungsmethoden}

Direkte Methoden betreffen direkt die Ernährung einer Person.  Sie lassen sich in zwei verschiedene Kategorien unterteilen; retrospektive Methoden und prospektive Methoden. Retrospektive Methoden sind Interviewmethoden und beziehen sich auf das zurückliegende Ernährungsverhalten einer Person.  Prospektive Methoden dagegen sind Protokollmethoden und erfassen den gegenwärtigen Verzehr. (Sichert, Oltersdorf et al. 1984)
Abb. Gliederung der Methoden zur Charakterisierung der Nahrungsaufnahme des Menschen (nach Sichert, Oltersdorf et al. 1984)

\subsubsection{Retrospektive Methoden}

Retrospektive Methoden beziehen sich auf den vergangenen Verzehr.  Je nach Methode bezieht sich die Erhebung auf einen kurzen Zeitraum (24-Stunden Erinnerungsprotokoll) oder längere Zeiträume von bis zu mehreren Monaten (Ernährungsgeschichte). Es gilt, dass der Nahrungsverzehr einer Person nach Art und Menge für den festgelegten Zeitraum so genau wie möglich erfasst wird.\cite{muller2007ernahrungsmedizinische}


\paragraph{24-Stunden Erinnerungsprotokoll}

Das 24-Stunden Erinnerungsprotokoll (24-hour recall) besteht aus einer Auflistung der Nahrungsaufnahme des Vortages bzw. der vergangenen 24 Stunden. Der Befragte wird gebeten, sich an alle Nahrungsmittel inklusive Mengen zu erinnern, die er in den letzten 24 Stunden aufgenommen hat. Die Durchführung gliedert sich in verschiedene Teilschritte. Zu Anfang erfolgt eine Befragung durch einen speziell geschulten Interviewer. Dieser erfragt, welche Lebensmittel in welchen Mengen vom Teilnehmer in den letzten 24 Stunden verzehrt wurden. Daraufhin werden die verzehrten Lebensmittel nach Art und Menge identifiziert. Es folgt eine Umrechnung der Mengenschätzungen in Gewichtseinheiten und schließlich die Berechnung der Inhaltsstoffe mittels Nährwerttabellen. Die Mengenangaben werden in haushaltsüblichen Größen (Tasse, Löffel, Stück) angegeben und erfasst. Um die Angaben von Portionsgrößen zu erleichtern, können Hilfsmittel wie Bilder oder Schablonen herangezogen werden. (Sichert et al. 1984) Weitere Faktoren, die Genauigkeit und Durchführung verbessern sollen, haben Wissenschaftler der ''Nordic Cooperation Group of Dietary Reasearchers" verfasst: NOCH ÄNDERN, WENN DAS BUCH DA IST!!!!!!

1. Die Studienteilnehmer dürfen von der kommenden Befragung nichts wissen, damit
sie ihre Essgewohnheiten nicht ändern.
2. Die Befragung erfolgt entweder persönlich oder am Telefon.
3. Das Interview soll in einer ruhigen und entspannten Atmosphäre ablaufen und dies
möglichst bei jedem Probanden gleich.
4. Die abgefragten Tage sollten sich über die ganze Woche verteilen, wenn das
Interview eines von mehreren ist.
5. Der Interviewer sollte stets mit dem ersten Getränk und der ersten Speise des Tages
beginnen. Bei Schichtarbeitern liegt der Zeitrahmen von Mitternacht bis Mitternacht.11
6. Es sollen neutrale und einfache Fragen gestellt werden. Wenn Lebensmittel erfragt
werden, die meist mit anderen kombiniert werden, sind diese zusätzlich abzufragen.
Beispiel: Brot wurde genannt – Welches Streichfett?
7. Hilfsmittel für die Abschätzung von Portionsgrößen, wie z.B. Fotos, Schablonen und
Modelle, sind erwünscht.
8. Eine vorgefertigte Lebensmittelliste mit zusätzlichem Freiraum für Besonderheiten
hilft bei der Protokollierung (Cameron, Staveren 1988).

\paragraph{Ernährungsgeschichte}

Bei der Ernährungsgeschichte werden allgemeine Ernährungsmuster und Ernährungsgewohnheiten erfragt. Eine geschulte Fachkraft erhebt im Dialog mit dem Teilnehmer die durchschnittliche Nahrungsaufnahme der letzten Wochen oder Monate. Es werden gezielt Fragen nach speziellen Verzehrsgewohnheiten und alltäglicher Nahrungsaufnahme gestellt. Die Befragung gliedert sich in mehrere Teilschritte. Zu Beginn werden die Verzehrsgewohnheiten in Zusammenhang mit der alltäglichen Nahrungsaufnahme in einem bestimmten Zeitraum erfragt. Der Teilnehmer wird zu seinem Gesundheitszustand und anderen ernährungsrelevanten Themen befragt. Der Interviewer erfasst Ernährungsgewohnheiten bei Haupt- und Zwischenmahlzeiten. Desweiteren muss er  individuelle Gewohnheiten berücksichtigen, wie z.B. Variation von Mahlzeiten, Art und Häufigkeit der Aufnahme bestimmter Lebensmittel, Portionsgrößen sowie saisonale Unterschiede liegt (Sichert et al. 1984). Hilfsmaterialien wie z.B. Bilder, Modelle, Haushaltsmaße und Nahrungsmittel werden eingesetzt, damit die Mengenangaben mehrr der Realität entsprechen (Oltersdorf 1981). Daraufhin werden die erfassten Daten in einer Gegenkontrolle überprüft. Mittels Formblatt mit Lebensmittelgruppen werden die Angaben zur Nahrungsaufnahme kontrolliert. Abschließend werden die Aufschreibungen ausgewertet, indem die ermittelte Nährstoffaufnahme mit Hilfe von Nährwerttabellen berechnet wird (Sichert et al. 1984). \cite{sichert1984ernaehrungs}

\paragraph{Verzehrshäufigkeitsfragebogen}

Der Verzehrshäufigkeitsfragebogen (Food-Frequency Questionnaire) dient der Erfassung von Ernährungsgewohnheiten oder -verhalten und ist im Rahmen epideminologischer Studien die am meisten eingesetzte Erhebungsmethode. \cite{kirch2006prävention} Die Methode wurde entworfen, um die alltägliche Ernährung zu beurteilen. Es wird die Häufigkeit bewertet, mit der Lebensmittel oder bestimmte Lebensmittelgruppen verzehrt werden. Die aufgeführten Lebensmittel sollen Hauptquelle einer bestimmten Nährstoffgruppe sein oder Lebensmittel, die üblicherweise vom Teilnehmer verzehrt werden. Die Häufigkeit der Nahrungsaufnahme wird durch ein Mutiple-Choice Raster erfasst, bei dem der Teilnehmer schätzt, wie häufig er ein bestimmtes Lebensmittel oder Getränk verzehrt. Wählen kann er aus Kategorien, die von ''nie'' oder ''weniger als einmal im Monat'' bis ''6 + pro Tag'' reichen. Enthalten die aufgeführten Lebensmittellisten bestimmte Portionsgrößen und sind entsprechend umfangreich, können quantitative Berechnungen von Energie- und Nährstoffaufnahme durchgeführt werden. Weniger ausführliche Verzehrhäufigkeitslisten dienen einer allgemeinen, qualitativen Einordnung des Ernährungsverhaltens. 

Abbildung selbstgemacht aus FFQ National Cancer Institute

\subsubsection{Prospektive Methoden}

Mit  prospektiver Ernährungserhebung wird der aktuelle Verzehr zum Erhebungszeitpunkt erfasst. Der Erhebungszeitraum ist begrenzt und beträgt meist drei, vier oder sieben aneinander folgende Tage. 

\paragraph{Verzehrsprotokoll}

Das Verzehrsprotokoll die Art und Menge der von ihnen aufgenommenen Lebensmittel sowie die Tageszeit des Verzehrs. Unterschieden wird zwischen dem Wiegeprotokoll und dem Schätzprotokoll. 
Das Wiegeprotokoll lässt sich wiederrum in die genaue Wiegemethode und die vereinfachte Wiegemethode einteilen \cite{sichert1984ernaehrungs}. Beim vereinfachten Wiegeprotokoll werden die verzehrsfertigen Portionen mit Hilfe einer Küchenwaage oder einem geeigneten Messgefäß abgemessen. Bei der genauen Wiegemethode wird die verzehrte Menge durch das Abwiegen jedes Lebensmittels bzw. Getränks vor und nach der Zubereitung sowie gegebenfalls entstehende Abfälle, Reste und nicht essbare Anteile bestimmt. \cite{PHN:587344} Das Schätzprotokoll basiert auf einer Abschätzung der aufgenommenen Menge entweder in Gramm oder in gewöhnlichen Haushaltsmaßen wie z.B. Esslöffel, Tasse, Scheibe.
Das Schätzprotokoll ist die in der ernährungsmedizinischen Praxis am häufigsten verwendete Methode. \cite{muller2007ernahrungsmedizinische}


\paragraph{Dublikatmethode}

Bei der Dublikatmethode (Doppelportionstechnik) werden die zu verzehrenden Lebensmittel in zweifacher Ausfertigung zubereitet. Die zweite Portion wird nicht verzehrt, sondern im Labor bezüglich Energie- und Nährstoffgehalt untersucht. 

\subsection{Computergestützte Ernährungserhebungsmethoden}

\subsubsection{Bilderfassung mittels Foto/Video}

\subsubsection{Pocket PC mit Barcodescanner}

\subsubsection{Mobile Methoden}

\section{Literaturanalyse}

\subsection{Erläuterung der Auswahl}

\subsection{Studien}
\subsubsection{Studien zu mobilen Methoden}
\subsubsection{Studien zu auditiven Methoden}
\subsubsection{Studien zu elektronischen Methoden}




