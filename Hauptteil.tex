% ----------------------------------------------------------------------------------------------------------
% Kapitel
% ----------------------------------------------------------------------------------------------------------
\section{Uebersicht über Ernährungserhebungsmethoden}

Im ersten Kapitel werden die bisher üblichen Ernährungerhebungsmethoden beschrieben. Unterteilt werden sie in indirekte und direkte Methoden. Die direkten Methoden werden nochmals in prospektive und retrospektive Methoden unterteilt. Das zweite Kapitel gibt einen Überblick über die bereits bestehenden computergestützen Ernährungserhebungsmethoden. Nachdem die verschiedenen Methoden vorgestellt wurden, werden die Ziele von Ernährungsmethoden beschrieben. Zuletzt werden allgemeine Probleme um Ernährungserhebungen aufgeführt. 

\subsection{Ziele von Ernährungserhebung}

Das Interesse, den Nahrungsverzehr der Bevölkerung zu erfassen, stieg ab Anfang des 19. Jahrhunderts an. Zunächst wurden mittels Daten aus der Nahrungsmittelproduktion und Bevölkerungszahlen Verbrauchsdaten gesammelt und erste Haushaltsrechnungen durchgeführt. Hintergrund war es, herauszufinden, wie viel Nahrung ein Arbeiter benötigt, um seine Arbeit aufrechterhalten zu können. 

\subsection{Traditionelle direkte Ernährungserhebungsmethoden}

Direkte Methoden betreffen direkt die Ernährung einer Person.  Sie lassen sich in zwei verschiedene Kategorien unterteilen; retrospektive Methoden und prospektive Methoden. Retrospektive Methoden sind Interviewmethoden und beziehen sich auf das zurückliegende Ernährungsverhalten einer Person.  Prospektive Methoden dagegen sind Protokollmethoden und erfassen das gegenwärtige Ernährungsverhalten. (Sichert, Oltersdorf et al. 1984)
Abb. Gliederung der Methoden zur Charakterisierung der Nahrungsaufnahme des Menschen (nach Sichert, Oltersdorf et al. 1984)

\subsubsection{Retrospektive Methoden}

Retrospektive Methoden beziehen sich auf den vergangenen Verzehr.  Die Befragung erfolgt entweder durch einen geschulten Interviewer oder Fragebögen. Je nach Methode, bezieht sich die Befragung auf einen kurzen Zeitraum (24 Stunden Erinnerungsprotokoll) oder längere Zeiträume von bis zu mehreren Monaten (Ernährungsgeschichte). In den folgenden Abschnitten werden drei gängige retrospektive Methoden beschrieben. 


\paragraph{24 Stunden Erinnerungsprotokoll}

Das 24 Stunden Erinnerungsprotokoll (24-hour recall) besteht aus einer Auflistung der Nahrungsaufnahme des Vortages bzw. der vergangenen 24 Stunden. Der Proband wird gebeten, sich an alle Nahrungsmittel inklusive deren Portionsgrößen zu erinnern, die er in den letzten 24 Stunden aufgenommen hat. 

\paragraph{Ernährungsgeschichte}

Bei der Ernährungsgeschichte erhebt eine geschulte Fachkraft die durchschnittliche Nahrungsaufnahme der letzten Wochen oder Monate. Die Erhebung erfolgt im Dialog mit dem Probanden. Es werden Lebensmittel und Portionsgrößen aufgelistet. 

\paragraph{Verzehrshäufigkeitsfragebogen}


\subsubsection{Prospektive Methoden}

\paragraph{Verzehrsprotokoll}

\paragraph{Dublikatmethode}

\paragraph{Beobachtungsmethoden}

\subsection{Computergestützte Ernährungserhebungsmethoden}

\subsubsection{Pocket PC mit Barcodescanner}

\subsubsection{Mobile Methoden}

\section{Literaturanalyse}

\subsection{Erläuterung der Auswahl}

\subsection{Studien}
\subsubsection{Studien zu mobilen Methoden}
\subsubsection{Studien zu auditiven Methoden}
\subsubsection{Studien zu elektronischen Methoden}




