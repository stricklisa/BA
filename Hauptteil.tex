% ----------------------------------------------------------------------------------------------------------
% Kapitel
% ----------------------------------------------------------------------------------------------------------





\section{Übersicht über Ernährungserhebungsmethoden}

Im ersten Kapitel werden die bisher üblichen Ernährungerhebungsmethoden beschrieben. Unterteilt werden sie in indirekte und direkte Methoden. Die direkten Methoden werden nochmals in prospektive und retrospektive Methoden unterteilt. Das zweite Kapitel gibt einen Überblick über die bereits bestehenden computergestützen Ernährungserhebungsmethoden. Nachdem die verschiedenen Methoden vorgestellt wurden, werden die Ziele von Ernährungsmethoden beschrieben. Zuletzt werden allgemeine Probleme um Ernährungserhebungen aufgeführt. 

\subsection{Ziele von Ernährungserhebung}
\subsection{Traditionelle direkte Ernährungserhebungsmethoden}

\subsubsection{Retrospektive Methoden}

\paragraph{24 Stunden Erinnerungsprotokoll}


\paragraph{Ernährungsgeschichte}


\paragraph{Verzehrshäufigkeitsfragebogen}


\subsubsection{Prospektive Methoden}

\paragraph{Verzehrsprotokoll}

\paragraph{Dublikatmethode}

\paragraph{Beobachtungsmethoden}

\subsection{Computergestützte Ernährungserhebungsmethoden}

\subsubsection{Pocket PC mit Barcodescanner}

\subsubsection{Mobile Methoden}

\section{Literaturanalyse}

\subsection{Erläuterung der Auswahl}

\subsection{Studien}
\subsubsection{Studien zu mobilen Methoden}
\subsubsection{Studien zu auditiven Methoden}
\subsubsection{Studien zu elektronischen Methoden}




