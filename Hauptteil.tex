% ----------------------------------------------------------------------------------------------------------
% Kapitel
% ----------------------------------------------------------------------------------------------------------
\section{Hauptteil}
Im ersten Kapitel werden die bisher üblichen Ernährungerhebungsmethoden beschrieben. Unterteilt werden sie in retrospektive und prospektive Methoden. Das zweite Kapitel gibt einen Überblick über die bereits bestehenden computergestützen Ernährungserhebungsmethoden. Nachdem die verschiedenen Methoden vorgestellt wurden, werden allgemeine Probleme um Ernährungserhebungen aufgeführt. 

\subsection{Übersicht über Ernährungserhebungsmethoden}
Lorem ipsum dolor sit amet, consetetur sadipscing elitr, sed diam nonumy eirmod tempor invidunt ut labore et dolore magna aliquyam erat, sed diam voluptua. At vero eos et accusam et justo duo dolores et ea rebum. Stet clita kasd gubergren, no sea takimata sanctus est Lorem ipsum dolor sit amet. Lorem ipsum dolor sit amet, consetetur sadipscing elitr, sed diam nonumy eirmod tempor invidunt ut labore et dolore magna aliquyam erat, sed diam voluptua. At vero eos et accusam et justo duo dolores et ea rebum. Stet clita kasd gubergren, no sea takimata sanctus est Lorem ipsum dolor sit amet.

\subsubsection{Indirekte Methoden}

\paragraph{Einkommens- und Verbrauchsstichproben}

\paragraph{Agrarstatistiken/Nahrungsbilanzen}	

\subsubsection{Direkte Methoden}

\paragraph{Retrospektive Methoden}

\paragraph{24 Stunden Erinnerungsprotokoll}


\paragraph{Diet History Recall}


\paragraph{Verzehrshäufigkeitsfragebogen (FFQ)}


\subsubsection{Prospektive Methoden}

\paragraph{Verzehrsprotokoll}

\paragraph{Dublikatmethode}

\paragraph{Beobachtungsmethoden}


\subsection{Computergestützte Ernährungserhebungsmethoden}

\paragraph{Pocket PC mit Barcodescanner}


\paragraph{Mobile Methoden}

\subsection{Ziele von Ernährungserhebung}

\subsection{Probleme um Ernährungerhebungen}

