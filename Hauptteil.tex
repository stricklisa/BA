% ----------------------------------------------------------------------------------------------------------
% Kapitel
% ----------------------------------------------------------------------------------------------------------
\section{Uebersicht über Ernährungserhebungsmethoden}

Ernärungserhebungsmethoden lassen sich in indirekte und direkte Methoden einteilen. Die indirekte Erfassung der Ernährung basiert auf der Sammlung statistischer Daten von Populationen oder Teilpopulationen, wie z.B. Agrarstatistiken, Einkommens- und Verbrauchsstatistik. Durch diese Zahlen können u.a. Ernährungssituationen verschiedener Länder gegenübergestellt werden \cite{muller2007ernahrungsmedizinische}. Direkte Methoden erfassen das individuelle Ernährungsverhalten einzelner Menschen. Sie werden nochmals in prospektive und retrospektive Methoden unterteilt. 


\subsection{Ziele von Ernährungserhebung}

Ernährungserhebungen werden durchgeführt, um den Lebensmittelverzehr einzelner oder mehrerer Personen zu erfassen. Wichtig ist die Erhebung im Gebiet der Ernährungsberatung, der Diagnose von Erkrankungen, Allergien und Unverträglichkeiten sowie für wissenschaftliche Studien. Durch die Erhebung und auswertung des Lebensmittelverzehrs können Zusammenhänge zwischen der Ernährung und Erkrankungen ermittelt werden. Das Bewusstsein über die Verbindung zwischen Ernährungsverhalten und der Entstehung von Krankheiten steigt stetig an. Die Ernährung gilt neben erhöhtem Blutdruck, hohen Blutfettwerten und Übergewicht zu den bedeutensten Risikofaktoren chronischer Erkrankungen \cite{hurrelmann2004einfuhrung}. Genaue Auskünfte über das Ernährungsverhalten sind daher ein wichtiger Faktor für Behandlung und Prävention. 

\subsection{Traditionelle direkte Ernährungserhebungsmethoden}

Direkte Methoden betreffen direkt die Ernährung einer Person.  Sie lassen sich in zwei verschiedene Kategorien unterteilen; retrospektive Methoden und prospektive Methoden. Retrospektive Methoden sind Interviewmethoden und beziehen sich auf das zurückliegende Ernährungsverhalten einer Person.  Prospektive Methoden dagegen sind Protokollmethoden und erfassen den gegenwärtigen Verzehr. (Sichert, Oltersdorf et al. 1984,)
Abb. Gliederung der Methoden zur Charakterisierung der Nahrungsaufnahme des Menschen (Sichert, Oltersdorf et al. 1984)

\subsubsection{Retrospektive Methoden}

Retrospektive Methoden beziehen sich auf den vergangenen Verzehr.  Je nach Methode bezieht sich die Erhebung auf einen kurzen Zeitraum (24-Stunden Erinnerungsprotokoll) oder längere Zeiträume von bis zu mehreren Monaten (Ernährungsgeschichte). Es gilt, dass der Nahrungsverzehr einer Person nach Art und Menge für den festgelegten Zeitraum so genau wie möglich erfasst wird.\cite{muller2007ernahrungsmedizinische}


\paragraph{24-Stunden Erinnerungsprotokoll}

Das 24-Stunden Erinnerungsprotokoll (24-hour recall) besteht aus einer Auflistung der Nahrungsaufnahme des Vortages bzw. der vergangenen 24 Stunden. Der Befragte wird gebeten, sich an alle Nahrungsmittel inklusive Mengen zu erinnern, die er in den letzten 24 Stunden aufgenommen hat. Die Durchführung gliedert sich in verschiedene Teilschritte. Zu Anfang erfolgt eine Befragung durch einen speziell geschulten Interviewer. Dieser erfragt, welche Lebensmittel in welchen Mengen vom Teilnehmer in den letzten 24 Stunden verzehrt wurden. Daraufhin werden die verzehrten Lebensmittel nach Art und Menge identifiziert. Es folgt eine Umrechnung der Mengenschätzungen in Gewichtseinheiten und schließlich die Berechnung der Inhaltsstoffe mittels Nährwerttabellen. Die Mengenangaben werden in haushaltsüblichen Größen (Tasse, Löffel, Stück) angegeben und erfasst. Um die Angaben von Portionsgrößen zu erleichtern, können Hilfsmittel wie Bilder oder Schablonen herangezogen werden. (Sichert et al. 1984) Weitere Faktoren, die Genauigkeit und Durchführung verbessern sollen, haben Wissenschaftler der ''Nordic Cooperation Group of Dietary Reasearchers" verfasst: NOCH ÄNDERN, WENN DAS BUCH DA IST!!!!!!

1. Die Studienteilnehmer dürfen von der kommenden Befragung nichts wissen, damit
sie ihre Essgewohnheiten nicht ändern.
2. Die Befragung erfolgt entweder persönlich oder am Telefon.
3. Das Interview soll in einer ruhigen und entspannten Atmosphäre ablaufen und dies
möglichst bei jedem Probanden gleich.
4. Die abgefragten Tage sollten sich über die ganze Woche verteilen, wenn das
Interview eines von mehreren ist.
5. Der Interviewer sollte stets mit dem ersten Getränk und der ersten Speise des Tages
beginnen. Bei Schichtarbeitern liegt der Zeitrahmen von Mitternacht bis Mitternacht.11
6. Es sollen neutrale und einfache Fragen gestellt werden. Wenn Lebensmittel erfragt
werden, die meist mit anderen kombiniert werden, sind diese zusätzlich abzufragen.
Beispiel: Brot wurde genannt – Welches Streichfett?
7. Hilfsmittel für die Abschätzung von Portionsgrößen, wie z.B. Fotos, Schablonen und
Modelle, sind erwünscht.
8. Eine vorgefertigte Lebensmittelliste mit zusätzlichem Freiraum für Besonderheiten
hilft bei der Protokollierung (Cameron, Staveren 1988).

\paragraph{Ernährungsgeschichte}

Bei der Ernährungsgeschichte werden allgemeine Ernährungsmuster und Ernährungsgewohnheiten erfragt. Eine geschulte Fachkraft erhebt im Dialog mit dem Teilnehmer die durchschnittliche Nahrungsaufnahme der letzten Wochen oder Monate. Es werden gezielt Fragen nach speziellen Verzehrsgewohnheiten und alltäglicher Nahrungsaufnahme gestellt. Die Befragung gliedert sich in mehrere Teilschritte. Zu Beginn werden die Verzehrsgewohnheiten in Zusammenhang mit der alltäglichen Nahrungsaufnahme in einem bestimmten Zeitraum erfragt. Der Teilnehmer wird zu seinem Gesundheitszustand und anderen ernährungsrelevanten Themen befragt. Der Interviewer erfasst Ernährungsgewohnheiten bei Haupt- und Zwischenmahlzeiten. Desweiteren muss er  individuelle Gewohnheiten berücksichtigen, wie z.B. Variation von Mahlzeiten, Art und Häufigkeit der Aufnahme bestimmter Lebensmittel, Portionsgrößen sowie saisonale Unterschiede liegt (Sichert et al. 1984). Hilfsmaterialien wie z.B. Bilder, Modelle, Haushaltsmaße und Nahrungsmittel werden eingesetzt, damit die Mengenangaben mehrr der Realität entsprechen (Oltersdorf 1981). Daraufhin werden die erfassten Daten in einer Gegenkontrolle überprüft. Mittels Formblatt mit Lebensmittelgruppen werden die Angaben zur Nahrungsaufnahme kontrolliert. Abschließend werden die Aufschreibungen ausgewertet, indem die ermittelte Nährstoffaufnahme mit Hilfe von Nährwerttabellen berechnet wird (Sichert et al. 1984). \cite{sichert1984ernaehrungs}

\paragraph{Verzehrshäufigkeitsfragebogen}

Der Verzehrshäufigkeitsfragebogen (Food-Frequency Questionnaire) dient der Erfassung von Ernährungsgewohnheiten oder -verhalten und ist im Rahmen epideminologischer Studien die am meisten eingesetzte Erhebungsmethode. \cite{kirch2006prävention} Die Methode wurde entworfen, um die alltägliche Ernährung zu beurteilen. Es wird die Häufigkeit bewertet, mit der Lebensmittel oder bestimmte Lebensmittelgruppen verzehrt werden. Die aufgeführten Lebensmittel sollen Hauptquelle einer bestimmten Nährstoffgruppe sein oder Lebensmittel, die üblicherweise vom Teilnehmer verzehrt werden. Die Häufigkeit der Nahrungsaufnahme wird durch ein Mutiple-Choice Raster erfasst, bei dem der Teilnehmer schätzt, wie häufig er ein bestimmtes Lebensmittel oder Getränk verzehrt. Wählen kann er aus Kategorien, die von ''nie'' oder ''weniger als einmal im Monat'' bis ''6 + pro Tag'' reichen. Enthalten die aufgeführten Lebensmittellisten bestimmte Portionsgrößen und sind entsprechend umfangreich, können quantitative Berechnungen von Energie- und Nährstoffaufnahme durchgeführt werden. Weniger ausführliche Verzehrhäufigkeitslisten dienen einer allgemeinen, qualitativen Einordnung des Ernährungsverhaltens. 

Abbildung selbstgemacht aus FFQ National Cancer Institute

\subsubsection{Prospektive Methoden}

Mit  prospektiver Ernährungserhebung wird der aktuelle Verzehr zum Erhebungszeitpunkt erfasst. Der Erhebungszeitraum ist begrenzt und beträgt meist drei, vier oder sieben aneinander folgende Tage. 

\paragraph{Verzehrsprotokoll}

Das Verzehrsprotokoll die Art und Menge der von ihnen aufgenommenen Lebensmittel sowie die Tageszeit des Verzehrs. Unterschieden wird zwischen dem Wiegeprotokoll und dem Schätzprotokoll. 
Das Wiegeprotokoll lässt sich wiederrum in die genaue Wiegemethode und die vereinfachte Wiegemethode einteilen \cite{sichert1984ernaehrungs}. Beim vereinfachten Wiegeprotokoll werden die verzehrsfertigen Portionen mit Hilfe einer Küchenwaage oder einem geeigneten Messgefäß abgemessen. Bei der genauen Wiegemethode wird die verzehrte Menge durch das Abwiegen jedes Lebensmittels bzw. Getränks vor und nach der Zubereitung sowie gegebenfalls entstehende Abfälle, Reste und nicht essbare Anteile bestimmt. \cite{PHN:587344} Das Schätzprotokoll basiert auf einer Abschätzung der aufgenommenen Menge entweder in Gramm oder in gewöhnlichen Haushaltsmaßen wie z.B. Esslöffel, Tasse, Scheibe.
Das Schätzprotokoll ist die in der ernährungsmedizinischen Praxis am häufigsten verwendete Methode. \cite{muller2007ernahrungsmedizinische}


\paragraph{Duplikatmethode}

Bei der Dupblikatmethode (Doppelportionstechnik) werden die zu verzehrenden Lebensmittel in zweifacher Ausfertigung zubereitet. Die zweite Portion wird nicht verzehrt, sondern im Labor bezüglich Energie- und Nährstoffgehalt untersucht. Entstehende Reste werden ebenfalls gewogen und vom Duplikat abgezogen. \cite{MethodenderErnaehrungs}

\newpage

\subsection{Computergestützte Ernährungserhebungsmethoden}

Computergestütze Erhebungsmethoden ermöglichen den Teilnhemern eine sofortige Eingabe der Ernährungsdaten. Nach Angaben der WHO hat sich seit 1980 die Zahl der Übergewichtigen weltweit verdoppelt. In 2008 waren mehr als 1,4 Milliarde Erwachsene übergewichtig. Davon über 200 Millionen Männer und fast 300 Millionen Frauen \cite{whoobesityfactsheet}. Mit wachsendem Bewusstsein für Übergewicht und Ernährungsverhalten steigt auch die Nutzung von Ernährungserhebungsmethoden rapide an.\\ Durch den technischen Fortschritt werden die tradiotionellen Methoden durch Applikationen (Apps) ersetzt, die entweder am Computer oder auf Mobiltelefonen eingesetzt werden können \cite{Morikawa:2012:FRS:2390776.2390779}. Vor allem die Verwendung von Mobiltelefonen ist in den letzten Jahren stark gestiegen. Im Jahr 2013 besitzen 90 Prozent aller Deutschen über 14 Jahren ein Mobiltelefon. Bei Senioren ab 65 Jahren sind es 68 Prozent und in den jüngeren Altersgruppen 97 Prozent (vgl. Bitkom 2013) \cite{63MillionenHandy} Smartphones sind Mobiltelefone, die neben den Basisdiensten Telefonie und Short Message Service (SMS) weitere Funktionen erfüllen. Übliche Zusatzdienste sind zum Beispiel Internetzugang, Terminkalender, audiovisuelle Aufnahme und Kamera- sowie Videofunktion. Aufgrund komplexer Betriebssysteme können Applikationen (Apps) installiert und für weitere Funktionen genutzt werden \cite{Wirtschaftslexikon}. In Deutschland sind 2013 40 Prozent der über 14 Jährigen Smartphone-Besitzer \cite{63MillionenHandy}. Durch die vielen Einsatzmöglichkeiten der Smartphones sind die Geräte im Alltag vieler Menschen verankert. Somit ergeben sich auch für die Ernährungserhebung neue Mittel. 

Die Entwicklung von Computerprogrammen zur Erfassung des Verzehrs vereinfacht die Ernährungserhebung bei einer großen Teilnehmerzahl. Der vom Robert-Koch-Institut durchgeführte Ernährungssurvey verwendete DISHES 98 (Diet Interview Software for Health Examination Studies). Es ist ein auf einem Softwarepaket basierendes Ernährungserhebungsprogramm, das den alltäglichen Verzehr der letzten vier Wochen erfasst und die computergestütze Weiterentwicklung des Ernährungstagebuches bildet.  Die Erhebung beginnt mit der Erfassung persönlicher Daten wie Alter, Geschlecht, Größe und Gewicht. Daraufhin werden die üblichen Ernährungsgewohnheiten, einschließlich  Verzehrshäufigkeit und Portionsgrößen, abgefragt. Die Teilnehmer wählen die Lebensmittel aus vorgegebenen Listen aus, nach nicht aufgeführten Lebensmittel kann gesucht werden. Um die Portionsgrößen zu bestimmen werden haushaltsübliche Modelle wie z.B. Tasse, Esslöffel oder Teller genutzt. Beendet wird das Interview mit Fragen zu u.a. Diäten, Nahrungsergänzungen und körperlicher Aktivität. \cite{validityofDISHES98}

\subsubsection{Automatisches, selbstverwaltetes 24-Stunden Protokoll}

Basierend auf dem traditionellen 24-Stunden Protokoll wurde eine Methode entwickelt, die die Erfassung der Ernährung durch eine webbasierte Applikation verbessern soll. Das automatische, selbstverwaltete 24-Studen Protokoll (Automated Self-Administered 24-Hour Dietary Recall, ASA24) basiert auf zwei Internetseiten - Der Teilnehmerseite und der Wissenschaftlerseite, welche Verwaltung von Logistik der Datensammlungen erlaubt und  Zugang zu Analyse-Datein gewährt. Die Teilnhemer füllen auf der für sie eingerichteten Internetseite das Protokoll aus. Sie werden mittels dynamischer Bedienungsmaske durch das Programm geleitet, wie z.B. Animationen, visuelle und auditive Hinweise. Die Teilnehmer werden gebeten, detaillierte Fragen zu Nahrungs- und Getränkeaufnahme und Zeit des Verzehrs. Weiterhin gibt es die optionale Möglichkeit, anzugeben, wo die Mahlzeiten eingenommen wurden, ob alleine oder in einer Gruppe gegessen wurde und ob währendessen der Fernseher oder Computer genutzt wurde. 
Die aufgenommenen Lebensmittel können in einer Suchfunktion aus einer Liste ausgewählt werden. Weitere Fragen zu Zubereitung und Portionsgröße sind zu beantworten. Um die Portionsgrößen leichter einschätzen zu können, werden Fotos zum Vergleich herangezogen. Deweiteren können Nahrungsergänzungsmittel angegeben werden.  \cite{asa24} \cite{Subar20121134}





\subsubsection{Fotographie Methode (Remote Food Photography Method)}

Die Fotographie Methode wurde entwickelt, um eine akkurate und zeitnahe Erfassung des Verzehrs zu ermöglichen. Dabei sollen die Teilnehmer in ihrem alltäglichen Essverhalten nicht eingeschränkt und die Portionsgrößen möglichst genau erfasst werden. \cite{5333123}
Die Teilnehmer benötigen ein mit einer Kamera- und Datentransferfunktion ausgestattetes mobiles Telefon. Es werden die zum Verzehr bestimmten Lebensmittel sowie die Reste fotografiert und die Bilder vom Teilnehmer zur Analyse verschickt. \cite{BJN:3324360}




\subsubsection{Pocket PC mit Barcodescanner}

\subsubsection{Mobile Methoden}

\section{Literaturanalyse}

\subsection{Erläuterung der Auswahl}

Die Auswahl der Studien erfolgte anhand verschiedener Auswahlkriterien. Es wurden die Datenbanken Pub Med, Google Scholar und ACM anhand der Stichpunkte ''nutrition'', ''mobile", "telephone" "dietary assessment" und Kombinationen daraus durchsucht. Die Suche wurde zeitlich auf einen Rahmen der letzten 10 Jahre beschränkt. Weiterhin wurden nur englisch oder deutschsprachige Studien verwendet. Aufgezeigt wurden 24 Studien. Davon ausgewählt wurden sieben Studien zu mobilen Telefonen und Studien zu Computerprogrammen. 

\subsection{Studien}
\subsubsection{Studien zu mobilen Methoden}
\subsubsection{Studien zu auditiven Methoden}
\subsubsection{Studien zu elektronischen Methoden}




