% ----------------------------------------------------------------------------------------------------------
% Kapitel
% ----------------------------------------------------------------------------------------------------------
\newpage
\section{Diskussion}

\textbf{Erhebungsmethoden, die Ernährung durch Computerprogramme erfassen}\\
Carter et al. zeigen, dass MMM als Ernährungserhebungsmethode grundsätzlich geeignet ist. Die Methode korreliert im Gruppendurchschnitt positiv mit den Berechnungen des 24-Stunden Protokolls. Die Ergebnisse zeigen, dass die Applikation My Meal Mate leichter in der Einhaltung der Erfassung, einfacher in der Benutzung und eine größere soziale Brauchbarkeit aufweist als traditionelle Erhebungsmethoden. Die Nutzer müssen sich nicht an den vergangenen Verzehr erinnern, da die Lebensmittel entweder sofort dokumentiert oder digital festgehalten werden. Nachteilig dabei ist, dass Nutzer die Lebensmittel manuell auswählen oder eingeben müssen. Dabei können Lebensmittel vergessen oder bewusst nicht angegeben werden. Weiterhin kommt es zu ungenauen Angaben der Portionsgrößen. Zwar sind Nutzer bei der Erfassung nicht mehr auf Papier und Stift angewiesen und können Lebensmittel aus einer großen Datenbank wählen,trotzdem bleibt das Problem der unpräzisen Erfassung bestehen. Die in der Studie hervorgehenden Ungenauigkeiten in der  Volumenschätzung weisen darauf hin, dass die App in diesem Bereich noch verbesserungswürdig ist. Ein weiterer Nachteil der Methode ist, dass sich die Nutzer an das Eingeben der Lebensmittel erinnern müssen. Zwar ist die App jederzeit zugänglich, jedoch muss zu jeder Mahlzeit das Mobiltelefon zu Händen sein. Ist der Akku des Geräts nicht geladen oder wurde es an einem anderen Ort vergessen, ist eine zeitnahe Erfassung nicht möglich. Zudem sind durch die manuelle Erfassung falsche Angaben und Underreporting möglich. Die Ergebnisse zeigen, dass die Applikation in der aktuellen Version die Dokumentation der Ernährung zwar erleichtert, diese jedoch nicht deutlich präziser erfasst als traditionelle Methoden.\\
In der Studie von Kirkpatrick et al. (2014) konnte nachgewiesen werden, dass ASA24 grundsätzlich Potential zu einer validen Ernährungserhebungsmethode hat. Ein großer Vorteil der Methode ist die Möglichkeit der  Ernährungserfassung einer großen Gruppe über mehrere Tage. Durch die Erhebung anhand eines kostenlosen, im Internet verfügbaren Programms ist kein Erhebungspersonal notwendig und die Nutzer werden bei der Eingabe ihrer Daten nicht durch diese beeinflusst. Somit sind sowohl der personelle Aufwand als auch die Kosten der Erhebung geringer als bei der traditionellen 24-Stunden Befragung am Telefon. Da eine Anwendung über einen längeren Zeitraum möglich ist, wird neben dem alltäglichen Verzehr auch das übliche Essverhalten erfasst. Zudem verbessert der Befragungsleitfaden die Genauigkeit der Angaben und reduziert die Zahl an Lebensmitteln, die bei der Eingabe vergessen werden. Für Personen, die im Umgang mit Computern und dem Internet geschult sind, sind die Durchführung und der zeitliche Aufwand der Erfassung gering. Jedoch zeigten Ettienne-Gittens et al. (2013) in ihrer Studie, dass Senioren die traditionelle 24-Stunden Befragung der computergestützten Version vorziehen. Aus der Studie ging außerdem hervor, dass ältere Probanden seltener einen Computer besitzen und weniger oft Zugang zum Internet haben. Zudem sind sie weniger geschult in der Nutzung von Computerprogrammen. Aus den Ergebnissen von Ettienne-Gittens et al. lässt sich also schließen, dass Methoden, die sich auf die Erfassung anhand von Computerprogrammen oder Mobiltelefonen stützen, für Senioren weniger gut geeignet sind als für jüngere Personen. Ein weiterer Nachteil computergestützter 24-Stunden Protokolle wie ASA24 ist, dass sie sich auf die eigenen Angaben der Nutzer stützen. Falsche Angaben der Teilnehmer wie Under- oder Overreporting sind möglich. Somit zeigt die Methode in diesem Punkt keine Verbesserung zum traditionellen 24-Stunden Protokoll. Weiterhin ist das Programm darauf angewiesen, dass Nutzer die Portionsgrößen selbst schätzen. Dies führt zu ungenauen Ergebnissen bei der Energie- und Nährstoffberechnung. Zudem gibt es das Programm bisher noch nicht als mobile Version, sodass die Nutzer zum Eintragen auf einen Computer bzw. Laptop angewiesen sind. \\
Beide Methoden sind in den jeweiligen Studien als valide bewertet worden. Sie eignen sich für Personen, die im Umgang mit Computer und Internet beziehungsweise Smartphone geschult sind und das entsprechende Gerät besitzen. Beide Methoden weisen Vorteile gegenüber der traditionellen 24-Stunden Befragung auf, wie zum Beispiel dem geringeren Kosten- und Personalaufwand. Bedeutende Nachteile wie der Ungenauigkeit bei eigener Portionsgrößenschätzung und der Möglichkeit von Misreporting aufgrund eigener Angaben bleiben weiterhin bestehen und verringern die Präzision der Ernährungserhebung. 

\textbf{Erhebungsmethoden, die Ernährung mittels Fotographie und manueller Analyse erfassen}\\
In der Studie von Gemming et al. (2013) konnte nachgewiesen werden, dass die Verwendung von tragbaren Kameras die Angaben beim 24-Stunden Protokoll verbessert. Ein wichtiger Vorteil der SenseCam ist, dass sich die Methode nicht auf die Angaben und das Erinnerungsvermögen der Nutzer stützt, da die auf den Fotos erfassten Lebensmittel Grundlage der Erhebung sind. Dabei zeigte sich, dass das Betrachten der Fotos zu einer vollständigeren Erfassung der verzehrten Lebensmittel führte und somit das Misreporting verringerte, was zu genaueren Ergebnissen der Energieaufnahme leitete. Insgesamt ist dies ein klarer Hinweis darauf, dass tragbare Kameras unerwünschtes Underreporting verringern könnten. Die Durchführung der Fotoerfassung gestaltet sich sehr einfach, da die Kamera automatisch Bilder aufnimmt. Die eigentliche Erfassung der verzehrten Lebensmittel ist dagegen zeitaufwendig, da zuerst die Fotos auf einen Computer übertragen werden, dann aus den vielen Aufnahmen geeignete Bilder von Essenssituationen rausgefiltert und schließlich gemeinsam mit dem Erhebungspersonal einzelne Lebensmittel sowie Portionsgrößen dokumentiert werden müssen. Vorteil dabei ist jedoch, dass die Angaben der Befragten nicht durch das Erhebungspersonal beeinflusst werden können. Ein weiterer Vorteil ist, dass Portionsgrößen mit Hilfe von Fotos genauer abschätzbar sind. Dieses Ergebnis unterstützen auch Williamson et al. (2003), die in ihrer Studie Portionsgrößen von gewogenen Lebensmitteln mit eigenen Schätzungen und Schätzungen mit Hilfe von Fotos verglichen haben. Ein Nachteil der Methode ist, dass das Ernährungsverhalten der Nutzer beeinflusst werden könnte. Der Grad, in dem tragbare Kameras die Nahrungsaufnahme und das Essverhalten beeinflussen, wurde in dieser Studie nicht bestimmt und sollte noch erforscht werden. Dadurch dass tragbare Kameras wie die SenseCam dauerhaft aktiv sind, zeichnen die Geräte nicht nur Essenssituationen auf, sondern sämtliche Aktivitäten. Das größte Problem dabei könnte sein, die richtige Balance zwischen der vollständigen Erfassung von Essenssituationen und der Privatsphäre der Nutzer zu finden. Weiterhin besteht ein hoher personeller, und somit auch finanzieller Aufwand, da die eigentliche Erfassung anhand der Fotos gemeinsam mit geschultem Personal geschieht. Um die Verwendung tragbarer Kameras zur  Ernährungserfassung zu validieren, sind noch weitere Studien mit größerer Teilnehmerzahl nötig. \\
Die Ergebnisse der Validitätsstudie von Martin et al. (2012) unterstützen, dass die Remote Food Photography Method eine zuverlässige Methode zur Messung der Energie- und Nährstoffzufuhr von Erwachsenen ist. Bedeutende Vorteile der Methode sind, dass erstens das Abschätzen von Portionsgrößen nicht durch die Nutzer und zweitens die Eingabe der Lebensmittel nicht manuell erfolgen, sondern durch geschultes Personal. Es ist wahrscheinlich, dass aufgrund dessen die berechnete Energieaufnahme nur sehr gering von den Ergebnissen der Referenzmethode abwich. Zu ähnlichen Ergebnissen gelangten Martin et al. in einer im Jahr 2009 durchgeführten Studie. Dabei zeigte sich, dass sich die Berechnungen von Portionsgrößen zwischen dem Abwiegen von Lebensmitteln und Schätzungen von geschultem Personal nur gering unterscheiden (Martin et al., 2009b). Zudem ist positiv, dass bei der Studie kein Undereating festgestellt werden konnte. Jedoch ist die Studiengröße nicht groß genug und die Studiendauer zu kurz, um daraus einen sicheren Vorteil ableiten zu können. Da Nutzer in regelmäßigen Abständen durch Erinnerungsnachrichten an das Aufzeichnen des Verzehrs erinnert werden, wird ein Vergessen der Erfassung verringert. Sowohl Teilnehmer dieser Studie als auch Probanden einer von Martin et al. im Jahr 2009 durchgeführten Studie bewerteten die Methode als sehr zufriedenstellend (Martin et al, 2009a). Der größte Nachteil der Methode sind die hohen Personalkosten, die durch Einsatz von geschultem Personal entstehen. Die Durchführung könnte Personen schwerfallen, die im Umgang mit Kamerahandys nicht geschult sind. Ein weiteres Problem könnte sein, dass der nötige Referenzmarker nicht zur Erfassung neben die Speisen gelegt wird und somit kein geeignetes Foto entsteht. Six et al. (2010) zeigten in ihrer Studie, dass Jugendliche zum Großteil in der Lage sind, sowohl alle Lebensmittel als auch den Referenzmarker auf den Fotos festzuhalten. Ob RFPM generell eine geeignete Methode für Kinder und Jugendliche darstellt, wird zurzeit noch erforscht (Martin et al., 2012). \\
Die Ergebnisse der Studien zu Erhebungsmethoden, die Ernährung mittels Fotographie und manueller Analyse erfassen, zeigen, dass diese Methoden grundlegende Vorteile gegenüber traditionellen Methoden aufweisen. Die Erfassung der Lebensmittel mittels Fotographie und die Auswertung anhand geschulter Personen ist vielversprechend für eine präzise Berechnung von Energie- und Nährstoffgehalt. Die damit verbundenen hohen Kosten limitieren die Methoden jedoch in ihrer Anwendbarkeit. 

\textbf{Erhebungsmethoden, die Ernährung automatisiert erfassen}\\
Die Studienergebnisse von Lee et al. (2012) deuten insgesamt darauf hin, dass sich der Mobile Device Food Record mit automatischer Portionsgrößenberechnungen gut zur Erfassung der Ernährung eignet. Die Ergebnisse der Volumenbestimmung sind präziser als Schätzungen von Nutzern selbst. Six et al. (2010) untersuchten die Akzeptanz der Methode und zeigten, dass sie von Jugendlichen als einfach in der Durchführung und mit geringem zeitlichen Aufwand verbunden bewertet wurde. Es ist kein Personal zur Erfassung und Auswertung erforderlich, was sich positiv auf den Kostenaufwand auswirkt. 
Nachteil der Methode ist, dass sich die Nutzer an das Aufzeichnen der Speisen erinnern müssen. Außerdem muss auch bei dieser Methode ein Referenzmarker genutzt werden, was zu den oben genannten Problemen führen kann. Weiterhin sind bei dieser Methode Misreporting und somit verfälschte Energie- und Nährstoffberechnungen möglich, wenn zu verzehrende Lebensmittel nicht fotografiert werden. Weiterhin sind zwar die Ergebnisse der Portionsgrößenberechnung genauer als Schätzungen von Nutzern selbst, jedoch weichen die Ergebnisse der automatischen Berechnung zum Teil enorm von der tatsächlichen Portionsgröße der Lebensmittel ab, sodass die automatische Portionsgrößenberechnung noch ausbaufähig erscheint. Darüber hinaus ist die Studiengröße nicht groß genug und die Studiendauer zu kurz, um daraus eine allgemeine Empfehlung ableiten zu können. \\
Jia et al. (2013) konnten zeigen, dass anhand des EButtons eine objektive und präzise Energie- und Nährstoffberechnung möglich ist. Die Methode weist aufgrund automatischer Identifikation von Lebensmitteln, Portionsgrößenberechnung sowie Energie- und Nährstoffberechnung wesentliche Vorteile gegenüber traditionellen Verfahren auf. Zudem wirkt es sich positiv auf den zeitlichen und personellen Aufwand aus, was wiederrum zu geringeren Kosten führt. 
Jedoch deuten die Studienergebnisse insgesamt darauf hin, dass es zwei kritische Probleme bei der automatisierten Ernährungserhebung gibt. Sowohl die genaue Portionsgrößenbestimmung als auch die richtige Identifikation der Lebensmittel gestalteten sich aufgrund der hohen Komplexität der Speisen schwierig. In vielen Fällen war es nicht möglich, ein Lebensmittel oder Getränk anhand eines Fotos richtig zu bestimmen, was sich negativ auf die Genauigkeit der berechneten Energie- und Nährstoffzufuhr auswirkte. Da sich der EButton bisher noch in der Prototypenphase befindet und weiterentwickelt wird, könnte er in Zukunft eine valide Ernährungserhebungsmethode darstellen.\\
Erhebungsmethoden die Ernährung automatisiert erfassen, weisen entscheidende Vorteile gegenüber traditionellen Erhebungsmethoden auf. Die in der Einführung der Arbeit genannten Kriterien einer geeigneten Erhebungsmethode werden durch automatisierte Erhebung am besten erfüllt. Diese Methoden erwiesen sich in den Studien als valide, sie sind prospektiv und für den Alltag geeignet. Die Studienergebnisse zeigen, dass sie einfach und schnell in der Handhabung sind und kostengünstig im Vergleich zu Methoden, die auf die Unterstützung von Personal angewiesen sind. Die Portionsgrößen werden präziser bestimmt als durch Schätzungen der Anwender. Des Weiteren wird ein Misreporting verringert, bzw. bei der Verwendung des EButtons bei richtiger Anwendung nahezu vermieden. 


\newpage
\section{Schlussbetrachtung/Ausblick}

In dieser Bachelorarbeit wurden die Möglichkeiten und Grenzen computergestützter Ernährungserhebungsmethoden präsentiert. Aufgrund der vorliegenden Daten und der in Kapitel 3 vorgestellten Studien eignen sich computergestützte Methoden grundsätzlich zur Erhebung der Ernährung und bringen entscheidende Vorteile gegenüber traditionellen Methoden. Traditionelle, retrospektive Erhebungsmethoden wie das 24-Stunden Erinnerungsprotokoll, die Ernährungsgeschichte und der Verzehrshäufigkeitsfragebogen verlassen sich auf das Erinnerungsvermögen der Befragten sowie deren Fähigkeit, Portionsgrößen richtig abzuschätzen. 
Traditionelle, prospektive Erhebungsmethoden wie das Verzehrsprotokoll und die Duplikatmethode haben den Vorteil, dass sich die Nutzer nicht auf ihr Gedächtnis verlassen müssen. Jedoch stützen sie sich auf die Angaben der Nutzer und sind sehr aufwendig in der Durchführung, was eine Belastung für die Nutzer darstellt. \\
Alle computergestützten Ernährungserhebungsmethoden haben den großen Vorteil, dass sich die Nutzer nicht an verzehrte Speisen erinnern müssen. Eine bedeutende Entwicklung und Vorteil automatisierter Methoden ist, dass ein Abschätzen der Portionsgrößen nicht notwendig ist und Lebensmittel nicht eigenständig erfasst werden müssen. Diese Fähigkeit ist bisher zwar nicht ausgereift, erscheint jedoch vielversprechend für die Zukunft. Weiterhin ist zur Erhebung kein Personal notwendig, was sich positiv auf den finanziellen Aufwand auswirkt. Ein weiterer Vorteil ist, dass der Außer-Haus-Verzehr bei den computergestützten Methoden dank der Mobilität der Geräte kein Problem mehr darstellt. Die Erfassung der Nahrungsaufnahme mit dem Smartphone verläuft genauso beiläufig und ortsungebunden wie das Verfassen von Textnachrichten. Somit könnte ein weiterentwickelter Mobile Device Food Record eine präzise Ernährungserhebungsmethode darstellen, die gut für jüngere Personen geeignet ist, die ein Smartphone besitzen und regelmäßig nutzen. Als Weiterentwicklung der Methode könnte anstatt des Referenzmarkers, wie beim EButton der Teller als Maß zur Portionsgrößenberechnung dienen und die Erhebung somit noch unkomplizierter gestalten. Regelmäßig verschickte Erinnerungsnachrichten zum Erfassen der Speisen, die bereits bei der RFPM eingesetzt werden, würden den mdFR zusätzlich verbessern. Nur die Gefahr des Misreportings bleibt bei dieser Methode weiterhin bestehen. \\
Für Menschen, die den Umgang mit Computern und Smartphones nicht gewohnt sind, stellt der EButton eine vielversprechende Ernährungserhebungsmethode dar. Die Funktionen des EButtons entsprechen grundsätzlich denen des mdFR mit dem Vorteil, dass das Gerät nicht aktiv bedient werden muss. Zudem werden durch die automatische Erfassung des Verzehrs keine Mahlzeiten oder einzelne Lebensmittel nicht dokumentiert. Ein weiterer Vorteil ist, dass kein Referenzmarker notwendig ist und die Durchführung somit noch einfacher. Der EButton eignet sich somit nicht nur gut für Senioren, sondern könnte auch zum Beispiel bei der Ernährungserfassung von Demenzkranken eingesetzt werden. \\
Es laufen zudem bereits weitere Studien, die Validität von computergestützten Ernährungserhebungen überprüfen. Wissenschaftler vom Pennington Biomedical Research Center untersuchten beispielsweise seit November 2012, ob sich durch die Remote Food Photography Method die Formulanahrung für Säuglinge berechnen lassen kann (N.N., 2014). Ein Wissenschaftlerteam des National Cancer Institute arbeitet aktuell an der Entwicklung von ASA24-Mobile, einer mobilen Version des automatisierten 24-Stunden Protokolls. (National Cancer Institute, 2014).\\
Die Möglichkeiten der computergestützten Ernährungserhebungsmethoden sind groß und bestehende Grenzen werden überschritten, sodass neuartige Methoden die traditionellen Verfahren in Zukunft ablösen werden.
\newpage
