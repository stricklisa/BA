% ----------------------------------------------------------------------------------------------------------
% Kapitel
% ----------------------------------------------------------------------------------------------------------
\newpage
\section{Diskussion}

\textbf{Erhebungsmethoden, die Ernährung durch Computerprogramme erfassen}
Carter et al. zeigen, dass MMM als Ernährungserhebungsmethode grundsätzlich geeignet ist. Die Methode korreliert im Gruppendurchschnitt positiv mit den Berechnungen des 24-Stunden Protokolls. Die Ergebnisse zeigen, dass die Applikation My Meal Mate leichter in der Einhaltung der Erfassung, einfach in der Benutzung, hohen Komfort und soziale Brauchbarkeit aufweist als traditionelle Erhebungsmethoden. Die Nutzer müssen sich nicht an den vergangenen Verzehr erinnern, da die Lebensmittel entweder sofort dokumentiert oder digital festgehalten werden, um anhand der gespeicherten Daten den Verzehr zu erfassen. Nachteilig dabei ist, dass Nutzer die Lebensmittel manuell auswählen oder eingeben müssen. Zwar sind sie dabei nicht mehr auf Papier und Stift angewiesen und können Lebensmittel aus einer großen Datenbank wählen. Trotzdem Zudem liegt die  Portionsgrößenschätzung in den Händen der Nutzer. Die in der Studie hervorgehenden Ungenauigkeiten in der  Volumenschätzung weisen darauf hin, dass die App in diesem Bereich noch verbesserungswürdig ist. Ein weiterer Nachteil der Methode ist, dass sich die Nutzer an das Eingeben der Lebensmittel erinnern müssen. Zwar ist die App jederzeit zugänglich, jedoch muss zu jeder Mahlzeit das Mobiltelefon zu Händen sein. Ist der Akku des Geräts nicht geladen oder es wurde an einem anderen Ort vergessen, ist eine zeitnahe Erfassung nicht möglich. Zudem sind durch die manuelle Erfassung falsche Angaben und Underreporting möglich. \\
In der Studie von Kirkpatrick et al. (2014) konnte nachgewiesen werden, dass ASA24 grundsätzlich Potential zu einer validen Ernährungserhebungsmethode hat. Ein großer Vorteil der Methode ist die Möglichkeit der  Ernährungserfassung einer großen Gruppe über mehrere Tage. Durch die Erhebung anhand eines kostenlosen, im Internet verfügbaren Programms ist kein Erhebungspersonal notwendig und die Nutzer werden bei der Eingabe ihrer Daten nicht durch diese beeinflusst. Somit ist sowohl der personelle Aufwand als auch die Kosten der Erhebung geringer als bei der traditionellen 24-Stunden Befragung am Telefon. Da eine Anwendung über einen längeren Zeitraum möglich ist, wird neben dem alltäglichen Verzehr auch das übliche Essverhalten erfasst. Zudem verbessert der Befragungsleitfaden die Genauigkeit der Angaben und reduziert die Zahl an Lebensmitteln, die bei der Eingabe vergessen werden.  Für Personen, die im Umgang mit Computern und dem Internet geschult sind, sind die Durchführung und der zeitliche Aufwand der Erfassung gering. Jedoch zeigten Ettienne-Gittens et al. (2013) in ihrer Studie, dass Senioren die traditionelle 24-Stunden Befragung der Erfassung anhand der computergestützten Version bevorzugen. Ältere Probanden besitzen seltener einen Computer und haben weniger oft Zugang zum Internet. Zudem sind sie weniger geschult in der Nutzung von Computerprogrammen. Somit ist die ASA24 nicht gut geeignet für diese Personengruppe. Aus den Ergebnissen von Ettienne-Gittens et al. lässt sich schließen, dass auch andere Methoden, die sich auf die Erfassung anhand von Computerprogrammen oder Mobiltelefonen stützen, für Senioren weniger gut geeignet sind als traditionelle Methoden. Ein weiterer Nachteil computergestützter 24-Stunden Protokolle wie ASA24 ist, dass sie sich auf die eigenen Angaben der Nutzer stützen. Falsche Angaben der Teilnehmer wie Under- oder Overreporting sind möglich. Somit zeigt die Methode in diesem Punkt keine Verbesserung zum traditionellen 24-Stunden Protokoll. Weiterhin ist das Programm darauf angewiesen, dass Nutzer die Portionsgrößen selbst schätzen. Zudem gibt es das Programm bisher noch nicht als mobile Version, sodass der Nutzer zum Eintragen auf einen Computer bzw. Laptop angewiesen ist. 
Beide Methoden sind in den jeweiligen Studien als valide bewertet worden. Sie eignen sich für Personen, die im Umgang mit Computer und Internet beziehungsweise Smartphone geschult sind und das entsprechende Gerät besitzen. Beide Methoden weisen Vorteile gegenüber der traditionellen 24-Stunden Befragung auf, wie zum Beispiel dem geringeren Kosten- und Personalaufwand. Bedeutende Nachteile wie der Ungenauigkeit bei eigener Portionsgrößenschätzung und der Möglichkeit von Misreporting aufgrund eigener Angaben bleiben weiterhin bestehen und verringern die Präzision der Ernährungserhebung. \\

\textbf{Erhebungsmethoden, die Ernährung mittels Fotographie und manueller Analyse erfassen}\\
In der Studie von Gemming et al. (2013) konnte nachgewiesen werden, dass die Verwendung von tragbaren Kameras die Angaben beim 24-Stunden Protokoll verbessert. Ein wichtiger Vorteil der SenseCam ist, dass sich die Methode nicht auf die Angaben und das Erinnerungsvermögen der Nutzer stützt. Die auf den Fotos erfassten Lebensmittel sind Grundlage der Erhebung. Dabei zeigte sich, dass das Betrachten der Fotos zu einer vollständigeren Erfassung der verzehrten Lebensmittel führte und somit das Misreporting verringerte, was zu genaueren Ergebnissen der Energieaufnahme führte. Insgesamt ist dies ein klarer Hinweis darauf, dass tragbare Kameras unerwünschtes Underreporting verringern könnten. Die Durchführung der Fotoerfassung gestaltet sich sehr einfach, da die Kamera automatisch Bilder aufnimmt. Die eigentliche Erfassung der verzehrten Lebensmittel ist dagegen zeitaufwendig, da zuerst die Fotos auf einen Computer übertragen werden, dann aus den vielen Aufnahmen geeignete Bilder von Essenssituationen rausgefiltert und schließlich gemeinsam mit dem Erhebungspersonal einzelne Lebensmittel sowie Portionsgrößen dokumentiert werden. Vorteil dabei ist jedoch, dass die Angaben der Befragten nicht durch das Erhebungspersonal beeinflusst werden können. Ein weiterer Vorteil ist, dass Portionsgrößen mit Hilfe von Fotos genauer abschätzbar sind. Dieses Ergebnis unterstützen auch Williamson et al. (2003), die in ihrer Studie Portionsgrößen von gewogenen Lebensmitteln mit eigenen Schätzungen und Schätzungen mit Hilfe von Fotos verglichen haben. Ein Nachteil der Methode ist, dass das Ernährungsverhalten der Nutzer beeinflusst werden könnte. Der Grad, zu dem tragbare Kameras die Nahrungsaufnahme und das Essverhalten beeinflussen, wurde in dieser Studie nicht bestimmt und sollte noch erforscht werden. Tragbare Kameras wie die SenseCam dauerhaft aktiv, sodass sich die Nutzer nicht an das Aufzeichnen des Verzehrs erinnern müssen. Jedoch zeichnen die Geräte nicht nur die Essenssituationen auf, sondern sämtliche Aktivitäten. Das größte Problem dabei könnte sein, die richtige Balance zwischen der vollständigen Erfassung von Essenssituationen und der Privatsphäre der Nutzer zu finden. Weiterhin besteht ein hoher personeller Aufwand, da die eigentliche Erfassung anhand der Fotos gemeinsam mit geschultem Personal geschieht. Aufgrund dessen ist eine große Stichprobenzahl nicht möglich. Um die Verwendung tragbarer Kameras in der Ernährungserfassung zu validieren, sind noch weitere Studien mit größerer Teilnehmerzahl nötig. 
Die Ergebnisse der Validitätsstudie von Martin et al. (2012) unterstützen, dass die Remote Food Photography Method eine zuverlässige Methode zur Messung der Energie- und Nährstoffzufuhr von Erwachsenen ist. Bedeutende Vorteile der Methode sind, dass das Abschätzen von Portionsgrößen nicht durch die Nutzer und die Eingabe der Lebensmittel nicht manuell erfolgen, sondern durch geschultes Personal. Es ist wahrscheinlich, dass aufgrund dessen die berechnete Energieaufnahme nur sehr gering von den Ergebnissen der Referenzmethode abweicht. Zu ganz ähnlichen Ergebnissen  gelangten Martin et al. in einer im Jahr 2009 durchgeführten Studie. Da zeigte sich, dass sich die Berechnungen von Portionsgrößen zwischen Abwiegen von Lebensmitteln und geschultem Personal nur gering unterscheiden. Zudem ist positiv, dass bei der Studie kein Undereating festgestellt werden kann. Jedoch ist die Studiengröße nicht groß genug und die Studiendauer zu kurz, um daraus einen sicheren Vorteil ableiten zu können. Bei der RFPM erhalten die Nutzer Feedback und Erinnerungsnachrichten direkt aufs Smartphone, sodass sie individuell angepasst in regelmäßigen Abständen an das Aufzeichnen des Verzehrs erinnert werden. Somit wird ein Vergessen der Erfassung verringert. Sowohl Teilnehmer dieser Studie als auch Probanden einer von Martin et al. im Jahr 2009 durchgeführten Studie bewerteten die Methode als sehr zufriedenstellend. Der größte Nachteil der Methode sind die hohen Personalkosten, die durch deren Einsatz zur manuellen Identifikation und Analyse der erfassten Lebensmittel entstehen. Zudem könnte Personen, die im Umgang mit Mobiltelefonen mit Kamerafunktion nicht geschult sind, die Durchführung schwer fallen. Somit scheint die Methode nicht geeignet für ältere Personengruppen. Ein weiteres Problem könnte sein, dass der nötige Referenzmarker nicht zur Erfassung neben die Speisen gelegt wird und somit kein geeignetes Foto entsteht. Six et al. (2010) zeigten in ihrer Studie, dass Jugendliche zum Großteil in der Lage sind, sowohl alle Lebensmittel als auch den Referenzmarker auf den Fotos festzuhalten. Ob RFPM generell eine geeignete Methode für Kinder und Jugendliche darstellt, wird zurzeit noch erforscht (Martin et al., 2012).
Die Ergebnisse der Studien zu Erhebungsmethoden, die Ernährung mittels Fotographie und manueller Analyse erfassen, zeigen, dass diese Methoden grundlegende Vorteile gegenüber traditionellen Methoden aufweisen. Die Erfassung der Lebensmittel mittels Fotographie und die Auswertung anhand von geschultem Personal scheinen vielversprechend für eine präzise Berechnung von Energie- und Nährstoffgehalt. Die damit verbundenen hohen Kosten limitieren die Methoden jedoch in ihrer Anwendbarkeit. 




\newpage
\section{Schlussbetrachtung/Ausblick}


In dieser Bachelorarbeit wurden die Möglichkeiten und Grenzen computergestützter Ernährungserhebungsmethoden präsentiert. Aufgrund der vorliegenden Daten und der in Kapitel 4 vorgestellten Studien eignen sich computergestützte Methoden grundsätzlich zur Erhebung der Ernährung. Traditionelle, retrospektive Erhebungsmethoden wie das 24-Stunden Erinnerungsprotokoll, die Ernährungsgeschichte und der Verzehrshäufigkeitsfragebogen verlassen sich auf das Erinnerungsvermögen der Befragten sowie deren Fähigkeit, Portionsgrößen richtig abzuschätzen. 
Traditionelle, prospektive Erhebungsmethoden wie das Verzehrsprotokoll und die Duplikatmethode haben den Vorteil, dass sich die Nutzer nicht auf ihr Gedächtnis verlassen müssen. Jedoch stützen sie sich auf die Angaben der Nutzer und sind sehr aufwendig in der Durchführung. 
Alle computergestützten Ernährungserhebungsmethoden haben den großen Vorteil, dass die Nutzer sich nicht an verzehrte Speisen erinnern müssen. Zudem weisen die verschiedenen Methoden spezielle Vor- und Nachteile auf. Ein bedeutender Vorteil einiger Methoden ist, dass ein Abschätzen der Portionsgrößen bei einigen Methoden nicht notwendig ist. Bei EButton und mdFR erfolgt die Volumenberechnung automatisch mittels  speziell entwickelter Softwares. Diese Fähigkeit ist bisher nicht zwar nicht ausgereift, scheint aber vielversprechend für die Zukunft zu sein. Zudem ist zur Erhebung kein Personal notwendig, was sich positiv auf den finanziellen Aufwand auswirkt. Ein weiterer Vorteil ist, dass das Erfassen der Nahrungsaufnahme mit dem Smartphone genauso beiläufig und ortsungebunden wie das Verfassen von Textnachrichten. Somit könnte ein weiterentwickelter Mobile Device Food Record eine präzise Ernährungserhebungsmethode darstellen, die gut für Personen geeignet ist, die ein Smartphone besitzen und regelmäßig nutzen. 
Für Senioren …

Es laufen bereits weitere Studien, um die Validität von computergestützten Ernährungserhebungen zu überprüfen. Wissenschaftler vom Pennington Biomedical Research Center überprüfen seit November 2012, ob sich durch die Remote Food Photography Method die Formulanahrung für Säuglinge berechnen lassen kann. Die Studie läuft unter dem Titel "Baby Bottle: Remote Food Photography Method in Infants" noch bis Dezember 2014 (N.N., 2014). Ein Wissenschaftlerteam des National Cancer Institute arbeitet aktuell an der Entwicklung von ASA24- Mobile, einer mobilen Version des automatisierten 24-Stunden Protokolls. Wann das Projekt beendet sein wird, ist noch nicht bekannt. (N.N., 2014l). Entwicklungen dieser Projekte werden die Genauigkeit bisher bestehender computergestützter Ernährungserhebungsmethoden weiter verbessern, deren Kosten reduzieren und die Belastung der Befragten minimieren.  


\newpage
