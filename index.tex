\documentclass[12pt,a4paper,bibliography=totocnumbered,listof=totocnumbered]{scrartcl}
\usepackage[ngerman]{babel}
\usepackage[utf8x]{inputenc}
\usepackage[T1]{fontenc}
\usepackage{lmodern}
\usepackage{amsmath}
\usepackage{amsfonts}
\usepackage{amssymb}
\usepackage{graphicx}
\usepackage{fancyhdr}
\usepackage{tabularx}
\usepackage{geometry}
\usepackage{setspace}
\usepackage[right]{eurosym}
\usepackage[printonlyused]{acronym}
\usepackage{subfig}
\usepackage{floatflt}
\usepackage[usenames,dvipsnames]{color}
\usepackage{colortbl}
\usepackage{paralist}
\usepackage{array}
\usepackage{titlesec}
\usepackage{parskip}
\usepackage[right]{eurosym}
\usepackage{picins}
\usepackage[subfigure,titles]{tocloft}
\usepackage[pdfpagelabels=true]{hyperref}

\usepackage{listings}
\lstset{basicstyle=\footnotesize, captionpos=b, breaklines=true, showstringspaces=false, tabsize=2, frame=lines, numbers=left, numberstyle=\tiny, xleftmargin=2em, framexleftmargin=2em}
\makeatletter
\def\l@lstlisting#1#2{\@dottedtocline{1}{0em}{1em}{\hspace{1,5em} Lst. #1}{#2}}
\makeatother

\geometry{a4paper, top=27mm, left=30mm, right=20mm, bottom=35mm, headsep=10mm, footskip=12mm}

\hypersetup{unicode=false, pdftoolbar=true, pdfmenubar=true, pdffitwindow=false, pdfstartview={FitH},
	pdftitle={Abschlussarbeit},
	pdfauthor={Lisa Düpre},
	pdfsubject={Abschlussarbeit},
	pdfcreator={\LaTeX\ with package \flqq hyperref\frqq},
	pdfproducer={pdfTeX \the\pdftexversion.\pdftexrevision},
	pdfkeywords={Abschlussarbeit},
	pdfnewwindow=true,
	colorlinks=true,linkcolor=black,citecolor=black,filecolor=magenta,urlcolor=black}
\pdfinfo{/CreationDate (D:20110620133321)}

% Paragraph soll das neue subsubsubsection werden
\makeatletter
\renewcommand\paragraph{\@startsection{paragraph}{4}{\z@}%
            {-2.5ex\@plus -1ex \@minus -.25ex}%
            {1.25ex \@plus .25ex}%
            {\normalfont\normalsize\bfseries}}
\makeatother
\setcounter{secnumdepth}{4} % how many sectioning levels to assign numbers to
\setcounter{tocdepth}{4}    % how many sectioning levels to show in ToC



\begin{document}

\titlespacing{\section}{0pt}{12pt plus 4pt minus 2pt}{-6pt plus 2pt minus 2pt}

% Kopf- und Fusszeile
\renewcommand{\sectionmark}[1]{\markright{#1}}
\renewcommand{\leftmark}{\rightmark}
\pagestyle{fancy}
\lhead{}
\chead{}
\rhead{\thesection\space\contentsname}
\lfoot{Beispiel für eine Abschlussarbeit\newline auch mit langem Titel}
\cfoot{}
\rfoot{\ \linebreak Seite \thepage}
\renewcommand{\headrulewidth}{0.4pt}
\renewcommand{\footrulewidth}{0.4pt}

% Vorspann
\renewcommand{\thesection}{\Roman{section}}
\renewcommand{\theHsection}{\Roman{section}}
\pagenumbering{Roman}

% ----------------------------------------------------------------------------------------------------------
% Titelseite
% ----------------------------------------------------------------------------------------------------------
\thispagestyle{empty}
\begin{center}
	\includegraphics[scale=.3]{Bilder/640px-Uni_Bonn.png}\\
	\vspace*{2cm}
	\Large
	\textbf{Landwirtschaftliche Fakultät}\\
	\vspace*{2cm}
	\Huge
	\textbf{Abschlussarbeit}\\
	\vspace*{0.5cm}
	\large
	über das Thema\\
	\vspace*{1cm}
	\textbf{Möglichkeiten und Grenzen von computergestützten Ernährungserhebungsmethoden}\\
	\vspace*{2cm}
	
	\vfill
	\normalsize
	\newcolumntype{x}[1]{>{\raggedleft\arraybackslash\hspace{0pt}}p{#1}}
	\begin{tabular}{x{6cm}p{7.5cm}}
		\rule{0mm}{5ex}\textbf{Autor:} & Lisa Düpre\\ 
		\rule{0mm}{5ex}\textbf{Prüfer:} & Prof. Dr. Monika Hartmann \\ 
		\rule{0mm}{5ex}\textbf{Abgabedatum:} & 01.01.2013 \\ 
	\end{tabular} 
\end{center}
\pagebreak



% ----------------------------------------------------------------------------------------------------------
% Verzeichnisse
% ----------------------------------------------------------------------------------------------------------
% TODO Typ vor Nummer
\renewcommand{\cfttabpresnum}{Tab. }
\renewcommand{\cftfigpresnum}{Abb. }
\settowidth{\cfttabnumwidth}{Abb. 10\quad}
\settowidth{\cftfignumwidth}{Abb. 10\quad}

\titlespacing{\section}{0pt}{12pt plus 4pt minus 2pt}{2pt plus 2pt minus 2pt}
\singlespacing
\rhead{INHALTSVERZEICHNIS}
\renewcommand{\contentsname}{Inhaltsverzeichnis}
\phantomsection
%\addcontentsline{toc}{section}{\texorpdfstring{I \hspace{0.35em}Inhaltsverzeichnis}{Inhaltsverzeichnis}}
%\addtocounter{section}{1}
\tableofcontents
\pagebreak
\rhead{VERZEICHNISSE}
\listoffigures
\pagebreak

% bei Bedarf eines Tabellenverzeichnis wieder reinnehmen
%\listoftables
%\pagebreak
%\renewcommand{\lstlistlistingname}{Listing-Verzeichnis}
%{\labelsep2cm\lstlistoflistings}
%\pagebreak

% ----------------------------------------------------------------------------------------------------------
% Abkürzungen
% ----------------------------------------------------------------------------------------------------------
\section{Abkürzungsverzeichnis}
\begin{acronym}[OSGi] % längste Abkürzung steht in eckigen Klammern
	\setlength{\itemsep}{-\parsep} % geringerer Zeilenabstand
	\acro{OSGi}{Open Service Gateway initiative}
\end{acronym}
\newpage

% ----------------------------------------------------------------------------------------------------------
% Inhalt
% ----------------------------------------------------------------------------------------------------------
% Abstände Überschrift
\titlespacing{\section}{0pt}{12pt plus 4pt minus 2pt}{-6pt plus 2pt minus 2pt}
\titlespacing{\subsection}{0pt}{12pt plus 4pt minus 2pt}{-6pt plus 2pt minus 2pt}
\titlespacing{\subsubsection}{0pt}{12pt plus 4pt minus 2pt}{-6pt plus 2pt minus 2pt}

% Kopfzeile
\renewcommand{\sectionmark}[1]{\markright{#1}}
\renewcommand{\subsectionmark}[1]{}
\renewcommand{\subsubsectionmark}[1]{}
\lhead{Kapitel \thesection}
\rhead{\rightmark}

\onehalfspacing
\renewcommand{\thesection}{\arabic{section}}
\renewcommand{\theHsection}{\arabic{section}}
\setcounter{section}{0}
\pagenumbering{arabic}
\setcounter{page}{1}

% ---------------------------- Hier kannst Du die Kapitel einfügen ---------------------

% ----------------------------------------------------------------------------------------------------------
% Einleitung
% ----------------------------------------------------------------------------------------------------------
\section{Einleitung}

\subsection{Problemstellung}
Die Ernährung eines Menschen oder einer Personengruppe zu erfassen ist schon lange ein wichtiges Instrument, um herauszufinden, wie sich die Menschen ernähren, was sie essen und wie sich das Essverhalten auf den Körper und die Gesundheit auswirkt. Nur wenn genau erfasst wird, wie das Ernährungsverhalten einer Person aussieht, können Rückschlüsse auf den Einfluss der Ernährung auf die Gesundheit und eventuelles Krankheitsgeschehen geschlossen werden. 
Die Erfassung des Ernährungsverhaltens wurde lange Zeit über verschiedene Methoden gemacht. Den Methoden gemein ist, dass sowohl Erfassung als auch Auswertung sehr zeitaufwendig sind, eine hohe Compliance der Probanden vorraussetzen und die Kosten hoch sind. Zudem treten häufig Ungenauigkeiten und Fehler auf, durch welche keine präzisen Ergebisse entstehen. 
Wird das Ernährungsverhalten von großen Gruppen erfasst, ist es ohne computergestütze Programme fast unmöglich, klare Ergebnisse in einem angemessenen Zeitraum zu erzielen. 
Um die Ernährungserhebung effizienter zu gestalten, gibt es seit einiger Zeit Methoden, die auf Computerprogramme zurückgreifen. Der Zuwachs an Technologie steigt rasch an, somit muss die Entwicklung von zeitgerechten Erhebungsmethoden schnell geschehen und stetig angepasst und verbessert werden. 

\subsection{Zielsetzung}

\subsection{Vorgehensweise}
Die Arbeit gibt einen Überblick über die traditionellen Methoden und den neuen Methoden. Aus der Vielzahl der bestehenden Methoden werden die geläufigstens vorgestellt und im Anschluss miteinander verglichen. Hervorgehen werden die Vor- und Nachteile der einzelnen Methoden sowie deren Anwendbarkeit. Desweiteren werden die computergestützten 
Methoden daraufhin überprüft, welche Grenzen sich für diese neuen Methoden zeigen und welche 	Möglichkeiten sich dadurch dür die Ernährungserhebung ergeben. 
\pagebreak
% ----------------------------------------------------------------------------------------------------------
% Kapitel
% ----------------------------------------------------------------------------------------------------------
\section{Uebersicht über Ernährungserhebungsmethoden}

Im ersten Kapitel werden die bisher üblichen Ernährungerhebungsmethoden beschrieben. Unterteilt werden sie in indirekte und direkte Methoden. Die direkten Methoden werden nochmals in prospektive und retrospektive Methoden unterteilt. Das zweite Kapitel gibt einen Überblick über die bereits bestehenden computergestützen Ernährungserhebungsmethoden. Nachdem die verschiedenen Methoden vorgestellt wurden, werden die Ziele von Ernährungsmethoden beschrieben. Zuletzt werden allgemeine Probleme um Ernährungserhebungen aufgeführt. 

\subsection{Ziele von Ernährungserhebung}

Das Interesse, den Nahrungsverzehr der Bevölkerung zu erfassen, stieg ab Anfang des 19. Jahrhunderts an. Zunächst wurden mittels Daten aus der Nahrungsmittelproduktion und Bevölkerungszahlen Verbrauchsdaten gesammelt und erste Haushaltsrechnungen durchgeführt. Hintergrund war es, herauszufinden, wie viel Nahrung ein Arbeiter benötigt, um seine Arbeit aufrechterhalten zu können. 

\subsection{Traditionelle direkte Ernährungserhebungsmethoden}

Direkte Methoden betreffen direkt die Ernährung einer Person.  Sie lassen sich in zwei verschiedene Kategorien unterteilen; retrospektive Methoden und prospektive Methoden. Retrospektive Methoden sind Interviewmethoden und beziehen sich auf das zurückliegende Ernährungsverhalten einer Person.  Prospektive Methoden dagegen sind Protokollmethoden und erfassen das gegenwärtige Ernährungsverhalten. (Sichert, Oltersdorf et al. 1984)
Abb. Gliederung der Methoden zur Charakterisierung der Nahrungsaufnahme des Menschen (nach Sichert, Oltersdorf et al. 1984)

\subsubsection{Retrospektive Methoden}

Retrospektive Methoden beziehen sich auf den vergangenen Verzehr.  Die Befragung erfolgt entweder durch einen geschulten Interviewer oder Fragebögen. Je nach Methode, bezieht sich die Befragung auf einen kurzen Zeitraum (24 Stunden Erinnerungsprotokoll) oder längere Zeiträume von bis zu mehreren Monaten (Ernährungsgeschichte). In den folgenden Abschnitten werden drei gängige retrospektive Methoden beschrieben. Es gilt, dass der Nahrungsverzehr einer Person nach Art und Menge für den festgelegten Zeitraum so genau wie möglich erfasst wird.\cite{muller2007ernahrungsmedizinische}


\paragraph{24 Stunden Erinnerungsprotokoll}

Das 24 Stunden Erinnerungsprotokoll (24-hour recall) besteht aus einer Auflistung der Nahrungsaufnahme des Vortages bzw. der vergangenen 24 Stunden. Der Befragte wird gebeten, sich an alle Nahrungsmittel inklusive deren Mengen zu erinnern, die er in den letzten 24 Stunden aufgenommen hat. Die Durchführung gliedert sich in verschiedene Teilschritte. Zu Anfang erfolgt eine Befragung durch einen speziell geschulten Interviewer. Dieser erfragt, welche Lebensmittel in welchen Mengen vom Teilnehmer in den letzten 24 Stunden verzehrt wurden. Daraufhin werden die verzehrten Lebensmittel nach Art und Menge identifiziert. Es folgt eine Umrechnung der Mengenschätzungen in Gewichtseinheiten und schließlich die Berechnung der Inhaltsstoffe mittels Nährwerttabellen. Die Mengenangaben werden in haushaltsüblichen Größen (Tasse, Löffel, Stück) angegeben und erfasst. Um die Angaben von Portionsgrößen zu erleichtern, können Hilfsmittel wie Bilder oder Schablonen herangezogen werden. (Sichert et al. 1984)

\paragraph{Ernährungsgeschichte}

Bei der Ernährungsgeschichte werden allgemeine Ernährungsmuster und Ernährungsgewohnheiten erfragt. Eine geschulte Fachkraft erhebt in einem Dialog mit dem Teilnehmer die durchschnittliche Nahrungsaufnahme der letzten Wochen oder Monate. Es werden gezielt Fragen nach speziellen Verzehrsgewohnheiten und alltäglicher Nahrungsaufnahme gestellt. Die Befragung gliedert sich meist in folgende Teilschritte: Als erstes werden die Verzehrsgewohnheiten in Zusammenhang mit der alltäglichen Nahrungsaufnahme in einem bestimmten Zeitraum erfragt. Der Teilnehmer wird zu seinem Gesundheitszustand und andere ernährungsrelevanten Themen befragt. Der Interviewer erfasst Ernährungsgewohnheiten bei Haupt- und Zwischenmahlzeiten, wobei ein Augenmerk auch auf  Variationen von Mahlzeiten, Art und Häufigkeit der Aufnahme bestimmter Lebensmittel, Portionsgrößen sowie saisonale Unterschiede liegt. Daraufhin werden die erfassten Daten in einer Gegenkontrolle überprüft. Mittels Formblatt mit Lebensmittelgruppen werden die Angaben zur Nahrungsaufnahme kontrolliert. Abschließend werden die Aufschreibungen ausgewertet, indem die ermittelte Nährstoffaufnahme mit Hilfe von Nährwerttabellen berechnet wird. 

\paragraph{Verzehrshäufigkeitsfragebogen}

Der Verzehrshäufigkeitsfragebogen (Food Frequency Questionnaire) dient der Erfassung von Ernährungsgewohnheiten oder -verhalten und ist im Rahmen epideminologischer Studien die am meisten eingesetzte Erhebungsmethode. \cite{kirch2006prävention} Die Methode wurde entworfen, um die alltägliche Ernährung zu beurteilen, indem die Häufigkeit beurteilt, mit der Lebensmittel oder bestimmte Lebensmittelgruppen verzehrt werden. Die aufgeführten Lebensmittel sollten Hauptquelle einer bestimmten Nährstoffgruppe sein oder Lebensmittel, die üblicherweise vom Teilnehmer verzehrt werden. Die Häufigkeit der Nahrungsaufnahme wird durch ein Mutiple-Choice Raster erfasst, bei dem der Teilnehmer schätzt, wie häufig er ein bestimmtes Lebensmittel oder Getränk verzehrt. Wählen kann er aus Kategorien, die von ''nie'' oder ''weniger als einmal im Monat'' bis ''6 + pro Tag'' reichen. Enthalten die aufgeführten Lebensmittellisten bestimmte Portionsgrößen und sind entsprechend umfangreich, können quantitative Berechnungen von Energie- und Nährstoffaufnahme durchgeführt werden. Weniger ausführliche Verzehrhäufigkeitslisten dienen einer allgemeinen, qualitativen Einordnung des Ernährungsverhaltens. 

Abbildung selbstgemacht aus FFQ National Cancer Institute

\subsubsection{Prospektive Methoden}

Mit  prospektiverErnährungserhebung wird der aktuelle Verzehr erfasst. Der Erhebungszeitraum ist begrenzt und beträgt meist sieben aneinander folgende Tage. Übliche Methoden  sind in den folgenden Kapiteln aufgeführt. 

\paragraph{Verzehrsprotokoll}

Das Verzehrsprotokoll die Art und Menge der von ihnen aufgenommenen Lebensmittel sowie die Tageszeit des Verzehrs. Unterschieden wird zwischen einem Wiegeprotokoll und einem Schätzprotokoll. Beim Wiegeprotokoll werden die zum Verzehr bestimmten Lebensmittel mit Hilfe einer Waage abgemessen. Wichtig dabei ist, dass eventuell entstehende Reste ebenfalls gewogen und von der Menge abgezogen werden. Das Schätzprotokoll basiert auf einer Abschätzung der aufgenommenen Menge mittels haushaltsüblichen Maßen wie Esslöffel, Tasse, Scheibe usw. 



\paragraph{Dublikatmethode}

\paragraph{Wiegeprotokoll}

\subsection{Computergestützte Ernährungserhebungsmethoden}

\paragraph{Bilderfassung mittels Foto/Video}

\subsubsection{Pocket PC mit Barcodescanner}

\subsubsection{Mobile Methoden}

\section{Literaturanalyse}

\subsection{Erläuterung der Auswahl}

\subsection{Studien}
\subsubsection{Studien zu mobilen Methoden}
\subsubsection{Studien zu auditiven Methoden}
\subsubsection{Studien zu elektronischen Methoden}





% ----------------------------------------------------------------------------------------------------------
% Kapitel
% ----------------------------------------------------------------------------------------------------------
\section{Diskussion}

\subsection{Vor- und Nachteile traditionieller direkter Ernährungserhebungsmethoden}

\subsection{Vor- und Nachteile computergestützter Ernährungserhebungsmethoden}

\subsection{Vergleich von üblichen und computergestützten Methoden}



\section{Schlussbetrachtung/Ausblick}


% ----------------------------------------------------------------------------------------------------------
% Literatur
% ----------------------------------------------------------------------------------------------------------
\renewcommand\refname{Literaturverzeichnis}
%\bibliographystyle{myalpha}
\pagebreak

% ----------------------------------------------------------------------------------------------------------
% Anhang
% ----------------------------------------------------------------------------------------------------------
\pagenumbering{Roman}
\setcounter{page}{1}
\lhead{Anhang \thesection}

\begin{appendix}
%\section*{Anhang}
\phantomsection
%\addcontentsline{toc}{section}{Anhang}
\addtocontents{toc}{\vspace{-0.5em}}

%\section{GUI}
%Ein toller Anhang.

%\subsection*{Screenshot}
%\label{app:screenshot}
%Unterkategorie, die nicht im Inhaltsverzeichnis auftaucht.

\end{appendix}


\newpage
\thispagestyle{empty}
\begin{center}
	\vspace*{5em}
	\huge\textbf{Erklärung}\\
\end{center}
\vspace{2em}
Ich versichere persönlich, dass ich die vorstehende Arbeit selbständig und ohne Benutzung anderer als der angegebenen Hilfsmittel angefertigt habe. Alle Stellen, die wörtlich oder sinngemäß aus Veröffentlichungen entnommen sind, wurden als solche kenntlich gemacht. Die Arbeit ist in gleicher oder ähnlicher Form oder auszugsweise im Rahmen einer anderen Prüfung noch nicht vorgelegt worden.

\vspace{2em}
\begin{minipage}{\linewidth}
	\begin{tabular}{p{15em}p{15em}}
	Bonn, den 27.10.2014
	\end{tabular}
\end{minipage}




\end{document}
