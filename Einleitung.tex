
% ----------------------------------------------------------------------------------------------------------
% Einleitung
% ----------------------------------------------------------------------------------------------------------
\section{Einleitung}
Die Ernährung eines Menschen oder einer Personengruppe zu erfassen ist schon lange ein wichtiges Instrument, um herauszufinden, wie sich die Menschen ernähren, was sie essen und wie sich das Essverhalten auf den Körper und die Gesundheit auswirkt. Nur wenn genau erfasst wird, wie das Ernährungsverhalten einer Person aussieht, können Rückschlüsse auf den Einfluss der Ernährung auf die Gesundheit und eventuelles Krankheitsgeschehen geschlossen werden. 
Die Erfassung des Ernährungsverhaltens wurde lange Zeit über verschiedene Methoden gemacht. Den Methoden gemein ist, dass sowohl Erfassung als auch Auswertung sehr zeitaufwendig sind, eine hohe Compliance der Probanden vorraussetzen und die Kosten hoch sind. Zudem treten häufig Ungenauigkeiten und Fehler auf, durch welche keine präzisen Ergebisse entstehen. 
Wird das Ernährungsverhalten von großen Gruppen erfasst, ist es ohne computergestütze Programme fast unmöglich, klare Ergebnisse in einem angemessenen Zeitraum zu erzielen. 
Um die Ernährungserhebung effizienter zu gestalten, gibt es seit einiger Zeit Methoden, die auf Computerprogramme zurückgreifen. Der Zuwachs an Technologie steigt rasch an, somit muss die Entwicklung von zeitgerechten Erhebungsmethoden schnell geschehen und stetig angepasst und verbessert werden. Die Arbeit gibt einen Überblick über die traditionellen Methoden und den neuen Methoden. Aus der Vielzahl der bestehenden Methoden werden die geläufigstens vorgestellt und im Anschluss miteinander verglichen. Hervorgehen werden die Vor- und Nachteile der einzelnen Methoden sowie deren Anwendbarkeit. Desweiteren werden die computergestützten 
Methoden daraufhin überprüft, welche Grenzen sich für diese neuen Methoden zeigen und welche Möglichkeiten sich dadurch dür die Ernährungserhebung ergeben. 

\subsection{Problemstellung}

\subsection{Zielsetzung}