
% ----------------------------------------------------------------------------------------------------------
% Einleitung
% ----------------------------------------------------------------------------------------------------------
\section{Einleitung}

\subsection{Problemstellung}
Die Erhebung der menschlichen  Ernährung ist schon lange ein wichtiges Werkzeug, um Informationen über das Essverhalten bestimmter Bevölkerungsgruppen zu erhalten und Zusammenhänge zwischen Ernährung und Gesundheit aufzudecken (Müller, 2007).
Um diese Ziele zu erreichen, muss das Ernährungsverhalten einer Person möglichst genau erfasst werden. Untersuchungen des Verzehrverhaltens bieten wertvolle Erkenntnisse, um Interventionsprogramme zur Prävention von chronischen Krankheiten zu gestalten (Six et al. 2010). Nach  Angaben der World Health Organization (WHO) hat sich weltweit seit 1980 die Zahl der Übergewichtigen verdoppelt. Im Jahr 2008 waren mehr als 1,4 Milliarden Erwachsene übergewichtig (davon über 200 Millionen Männer und fast 300 Millionen Frauen) (vgl. WHO). Die Folgen von Fehlernährung sind schwere gesundheitliche Probleme wie Diabetes, Schlaganfall und Herzerkrankungen sowie hohe Kosten im Gesundheitswesen (Wirth und Hauner, 2013). Die auf die anhaltende Zunahme von Übergewicht und Adipositas zurückzuführenden Ausgaben führen zu einem erhöhten Interesse, praktische und neue Technologien zur Prävention der Adipositas zu erforschen (Ershow et al., 2004). Um das Essverhalten zu registrieren, wurden verschiedene Methoden entwickelt, die sich je nach Ziel und Umfang der Erfassung unterscheiden. Aus den erhobenen Informationen werden mittels verschiedener Datenbanken Nährstoff- und Energiezufuhr berechnet. Anhand der Ergebnisse können Rückschlüsse auf den Ernährungsstatus eines Menschen geschlossen werden. Jedoch sind Durchführung und Auswertung der meisten traditionellen Methoden wie dem Verzehrsprotokoll sehr zeitaufwendig und setzen eine hohe Folgebereitschaft der Nutzer voraus. Zudem sind sie oft kostenintensiv und es kommt häufig zu Problemen und Fehlern, vor allem durch falsches Schätzen der Portionsgrößen, Underreporting oder mangelndem Erinnerungsvermögen der Probanden. Somit entstehen ungenaue Ergebnisse, die zu falschen Beurteilungen und Rückschlüssen führen können (Schneider und Heseker, 2006).


Um die Ernährungserhebung zuverlässiger zu gestalten und besonders vor dem Hintergrund der stetig wachsenden Zahl ernährungsbedingter Krankheiten, ist es wichtig, Ernährungserhebungsmethoden zu entwickeln, die 
\begin{itemize}
\item valide,
\item prospektiv,
\item für den Alltag geeignet,
\item einfach und schnell in der Handhabung,
\item kostengünstig,
\item möglichst fehlerfrei sind.
\end{itemize}

Weiterhin sollen sie vor allem
\begin{itemize}
\item Portionsgrößen präzise berechnen
\item Underreporting vermeiden (Kovatsch, 2014).
\end{itemize}

Mit dem technologischen Fortschritt in vielen Bereichen des Gesundheitswesens, wurden in den letzten Jahren Methoden zur Ernährungserhebung entwickelt, die auf Computerprogramme zurückgreifen. Diese sollen die Erhebung effizienter gestalten. 


\subsection{Zielsetzung}

Durch die Einführung von computergestützten Methoden, stellt sich die Frage, ob und wie gut die bisher entwickelten Verfahren die in Kapitel 1.1 genannten Kriterien erfüllen. Zudem ergeben sich mit den neuen Methoden auch neue Probleme, wie zum Beispiel die Tatsache, dass sie für den Umgang mit Computern oder anderen elektronischen Geräten  ein Grundmaß an Erfahrung erfordern (Kirkpatrick et al.,2014). Welche bisher bestehenden Probleme gelöst werden und welche Schwierigkeiten mit der Einführung entstehen, wird in dieser Bachelorarbeit beschrieben. Weiterhin wird aufgezeigt, für welche Zielgruppen die verschiedenen Methoden geeignet sind.
Somit werden die Möglichkeiten und Grenzen von computergestützten Ernährungserhebungsmethoden aufgezeigt und die Anwendbarkeit dieser Methoden beurteilt. 

\subsection{Vorgehensweise}

Die Arbeit ist in folgende fünf Hauptkapitel gegliedert. Im \textbf{ersten Kapitel} findet eine Heranführung an das Thema dieser wissenschaftlichen Arbeit statt, indem die Problemstellung und die Zielsetzung genau definiert werden. Das \textbf{zweite Kapitel} beleuchtet zunächst allgemein Ziele von Ernährungserhebungen. Es folgt ein Überblick über die traditionellen Methoden, die in der Erfassung der Ernährung angewandt werden. Dabei werden die gängigsten direkten Methoden beschrieben. Daraufhin werden aktuelle, computergestützte Methoden vorgestellt. Es wird ein Überblick über die bisher entwickelten Methoden gegeben und sechs Methoden näher beschrieben. Im Anschluss werden im \textbf{dritten Kapitel} ausgewählte Studien und deren Ergebnisse beschrieben, welche die aufgeführten Erhebungsmethoden prüfen.  Im \textbf{vierten Kapitel} werden die Studienergebnisse diskutiert. Aus diesem Kapitel gehen die Vor- und Nachteile der einzelnen Methoden hervor. Schließlich bildet das \textbf{fünfte Kapitel} den inhaltlichen Abschluss der Bachelorarbeit. Es beinhaltet in Form einer Empfehlung ein Fazit und einen Ausblick auf die neusten Methoden.


