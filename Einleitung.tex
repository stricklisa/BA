
% ----------------------------------------------------------------------------------------------------------
% Einleitung
% ----------------------------------------------------------------------------------------------------------
\section{Einleitung}

\subsection{Problemstellung}
Die Ernährung der Menschen zu erheben ist schon lange ein wichtiges Werkzeug, um Informationen über das Ernährungsverhalten bestimmter Bevölkerungsgruppen zu erhalten sowie Zusammenhänge zwischen Ernährung und Gesundheit aufzudecken.\cite{muller2007ernahrungsmedizinische}
Um diese Ziele zu erreichen, muss das Ernährungsverhalten einer Person möglichst genau erfasst werden. Nur so können Rückschlüsse auf den Einfluss der Ernährung auf die Gesundheit und eventuelles Krankheitsgeschehen gechlossen werden. \\
Um das Essverhalten zu erfassen, wurden verschiedene Methoden entwickelt, die sich je nach Ziel und Umfang der Erfassung unterscheiden. Aus den erhobenen Informationen werden mittels verschiedener Datenbanken Nährstoff- und Energiezufuhr berechnet. Anhand der Ergebnisse können Rückschlüsse auf den Ernährungsstatus eines Menschen geschlossen werden. Jedoch sind Durchführung und Auswertung der bisher üblichen Methoden sehr zeitaufwendig, setzen eine hohe Compliance der Probanden voraus, sind kostenintensiv und es kommt oft zu Problemen und Fehlern, z.B. durch falsches Schätzen der Portionsgrößen, Underreporting oder mangelndem Erinnerungsvermögen der Probanden. Somit entstehen  ungenaue Ergebnisse, die zu falschen Beurteilungen und Rückschlüssen führen können.\cite{schneider20062}

Um die Ernährungserhebung zuverlässiger zu gestalten und besonders vor dem Hintergrund der stetig steigenden Zahl ernährungsbedingter Krankheiten, ist es wichtig, Ernährungserhebungsmethoden zu entwickeln, die präzise sind, der Zeit angepasst, einfach und schnell in der Handhabung und kostengünstig.\cite{kovatsch2014gesunde} Die Methoden sollen mögliche Ungenauigkeiten und Probleme soweit es geht ausschalten. \\
Mit dem Fortschritt der Technologie in vielen Bereichen des Gesundheitswesens, wurden in den letzten Jahren Methoden zur Ernährungserhebung entwickelt, die auf Computerprogramme zurückgreifen. Die Methoden sollen die Ernährungserhebung effizienter gestalten. \\
 



\subsection{Zielsetzung}


Durch die Einführung der computergestützten Methoden, stellt sich die Frage, ob und wie gut die bisher entwickelten Verfahren die oben genannten Kriterien erfüllen. Möglicherweise ergeben sich dadurch andere Probleme. Zum Beispiel erfordern die Methoden ein Grundmaß an Erfahrung im Umgang mit Computern oder anderen elektronischen Geräten. Welche bisher bestehenden Ungenauigkeiten gelöst werden, welche Probleme mit der Einführung entstehen und wie dort Abhilfe geschaffen werden kann, wird in dieser Bachelorarbeit behandelt. Es wird aufgezeigt, für welche Zielgruppen die verschiedenen Methoden geeignet sind.\\
Somit ist das Ziel der vorliegenden Arbeit, aufzuzeigen, welche verlässlichen computergestützten Ernährungserhebungsmethoden bereits auf dem Markt erschienen sind. Desweiteren soll die Anwendbarkeit dieser Methoden beurteilt werden. \\

Die Arbeit ist in folgende sechs Hauptkapitel untergliedert:\\
Im \textbf{ersten Kapitel} wird an das Thema dieser wissenschaftlichen Arbeit herangeführt, die Problemstellung gegeben und die Zielsetzung dargestellt. \\
Im \textbf{zweiten Kapitel} werden zunächst allgemein Ziele von Ernährungserhebung beschrieben. Es folgt ein Überblick über die bisherigen Methoden, die in der Erfassung der Ernährung angewandt werden. Dabei werden die gängigsten indirekten sowie direkten Methoden beschrieben. Anschließend werden verschiedene Probleme bei der Erhebung dargestellt.\\
Im \textbf{dritten Kapitel} werden aktuelle, computergestützte Methoden vorgestellt. Es wird einen Überblick über die bestehenden Methoden gegeben und einzelne Methoden näher beschrieben. \\
Im Anschluss werden im \textbf{vierten Kapitel} ausgewählte Studien und deren Ergebnisse beschrieben, die zu den Ausführungen im darauf folgenden Kapitel herangezogen wurden. \\
Im \textbf{fünften Kapitel} werden sowohl bisher übliche Methoden als auch computergestützte Methoden untereinander verglichen. In diesem Kapitel werden die Vor- und Nachteile der einzelnen Methoden hervorgehen. \\
Schließlich bildet das \textbf{sechste Kapitel} den inhaltlichen Abschluss der Bachelorarbeit. Es beinhaltet in Form einer Empfehlung ein Fazit und einen Ausblick in die Zukunft.

Es folgt ein Quellen- und Anhangsverzeichnis. 

