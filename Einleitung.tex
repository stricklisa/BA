
% ----------------------------------------------------------------------------------------------------------
% Einleitung
% ----------------------------------------------------------------------------------------------------------
\section{Einleitung}

\subsection{Problemstellung}
Die Ernährung eines Menschen oder einer Personengruppe zu erfassen ist schon lange ein wichtiges Instrument, um herauszufinden, wie sich die Menschen ernähren, was sie essen und wie sich das Essverhalten auf den Körper und die Gesundheit auswirkt. Nur wenn genau erfasst wird, wie das Ernährungsverhalten einer Person aussieht, können Rückschlüsse auf den Einfluss der Ernährung auf die Gesundheit und eventuelles Krankheitsgeschehen geschlossen werden. 
Die Erfassung des Ernährungsverhaltens wurde lange Zeit über verschiedene Methoden gemacht. Den Methoden gemein ist, dass sowohl Erfassung als auch Auswertung sehr zeitaufwendig sind, eine hohe Compliance der Probanden vorraussetzen und die Kosten hoch sind. Zudem treten häufig Ungenauigkeiten und Fehler auf, durch welche keine präzisen Ergebisse entstehen. 
Wird das Ernährungsverhalten von großen Gruppen erfasst, ist es ohne computergestütze Programme fast unmöglich, klare Ergebnisse in einem angemessenen Zeitraum zu erzielen. 
Um die Ernährungserhebung effizienter zu gestalten, gibt es seit einiger Zeit Methoden, die auf Computerprogramme zurückgreifen. Der Zuwachs an Technologie steigt rasch an, somit muss die Entwicklung von zeitgerechten Erhebungsmethoden schnell geschehen und stetig angepasst und verbessert werden. 
Gerade für den Hintergrund der immer steigenden Zahl an ernährungsbedingter Krankheiten und der wachsenden Erkenntnis, dass die Ernährung einen großen Einfluss auf die Gesundheit hat, ist es wichtig, Ernährungserhebungsmethoden zu entwickeln, die präzise sind, einfach und schnell in der Handhabung und kostengünstig. Die Methoden sollten mögliche Ungenauigkeiten und Probleme der genauen Erhebung so weit es geht ausschalten. 

\subsection{Zielsetzung}

Ziel der vorliegenden Arbeit ist es, einen Überblick über computergestütze Ernährungserhebungmethoden zu geben und aufzuzeigen, welche Möglichkeiten und Grenzen die vorhandenen Methoden aufweisen. 

\subsection{Vorgehensweise}
Im Hauptteil der vorliegenden Arbeit wird zunächst ein Überblick über die bisherigen Methoden der Ernährungserhebung gegeben. Dabei werden die fünf gängigsten Methoden beschrieben. Daraufhin werden neue, computergestützte Methoden vorgestellt. Es wird ein Überblick über die bestehenden Methoden gegeben und einzelne Methoden näher beschrieben. Im Anschluss werden sowohl alte und neue Methoden sowie die somputergestützten Methoden untereinander verglichen. Hervorgehen werden die Vor- und Nachteile der einzelnen Methoden sowie deren Anwendbarkeit. Desweiteren werden die computergestützten Methoden daraufhin überprüft, welche Grenzen sich für diese Methoden ergeben und welche Möglichkeiten sich dadurch dür die Ernährungserhebung ergeben. 